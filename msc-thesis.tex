\documentclass[a4paper,capchap,espacoduplo,normaltoc]{abntepusp}

\usepackage[bookmarks,colorlinks=true,citecolor=black,urlcolor=blue,linkcolor=black,pdfpagemode=UseNone]{hyperref}
\usepackage[centertags]{amsmath}
\usepackage{amsfonts}
\usepackage{amssymb}
\usepackage{amsthm}
\usepackage[T1]{fontenc}
\usepackage[utf8]{inputenc}
\usepackage[brazil]{babel}
\usepackage[alf,abnt-repeated-author-omit=yes]{abntex2cite}
\usepackage{url}
\usepackage{underscore}
\usepackage{txfonts}
\usepackage[ddmmyyyy]{datetime}
\usepackage{enumitem}
\usepackage{comment}

\usepackage{tikz}
\usepackage{adjustbox}
\usepackage{color}
\usepackage{graphicx}
\usepackage{pdfpages}
\usepackage{morefloats}
\usepackage{tabularx}
\usepackage[table]{colortbl}
\usepackage{multirow}
\usepackage{mathtools}
\usepackage{array}
\usepackage{pdfpages}
\usepackage{mathrsfs}

%para as imagens
\newcommand{\xfig}{\includegraphics[scale=0.007]{x.png}}
\newcommand{\cfig}{\includegraphics[scale=0.015]{check.png}}
\newcommand{\rot}[1]{\rotatebox{90}{#1}}
\newcommand{\urls}[1]{{\footnotesize{\url{#1}}}}

\graphicspath{ {images/} }

\usetikzlibrary{matrix,shapes,arrows,positioning}

%comments
\newcommand{\marcos}[1]{~~\textcolor{blue} {\textbf{MARCOS: #1}}}
\newcommand{\comments}[1]{~~\textcolor{cyan} {\textbf{#1}}}
\newcommand{\TODO}[1]{~~\textcolor{red} {\textbf{TODO: #1}}}

\newcommand{\fancyname}{Dizang }
\newcommand{\fancynameX}{\fancyname}

\newcolumntype{C}[1]{>{\centering\let\newline\\\arraybackslash\hspace{0pt}}m{#1}}
\newcolumntype{L}{>{\centering\arraybackslash}m{0,75cm}}
\newcolumntype{M}{>{\RaggedLeft\arraybackslash}m{3cm}}


\hyphenation{
e-le-va-da
U-ni-ver-si-da-de
wit-ness-in-dis-tin-guish-a-bil-i-ty
ze-ro-knowl-edge
non-triv-i-al
struc-ture-pre-serv-ing
%Diffie-Hellman?
}

%misc
%\newcommand{\samples}{\stackrel{\,_\$}{\gets}}
%\newcommand{\iseq}{\stackrel{\,_?}{=}}

%\newcommand{\itab}{\hspace{1em}}

\newcommand{\hash}{\mathcal{H}}
\newcommand{\bigO}{\mathcal{O}}
%\newcommand{\set}{\mathcal{S}}
%\newcommand{\algorithm}{\mathcal{F}}

%groups and fields
%\newcommand{\F}[1][]{\ifthenelse{\equal{#1}{}}{\mathbb{F}}{\mathbb{F}_{#1}}}
%\newcommand{\G}[1][]{\ifthenelse{\equal{#1}{}}{\mathbb{G}}{\mathbb{G}_{#1}}}
%\newcommand{\Zq}{\mathbb{Z}_q}
%\newcommand{\0}{\mathcal{O}}

%challenge-response protocol
%\newcommand{\setup}{\mathcal{G}}
%\newcommand{\prover}{\mathcal{P}}
%\newcommand{\verifier}{\mathcal{V}}
%\newcommand{\eval}{\mathcal{F}}

%gs proof
%\newcommand{\crs}{\varsigma}
%\newcommand{\trapdoor}{\tau} %maybe I will not use this
%\newcommand{\comm}{\kappa}
%\newcommand{\pok}{\phi}

%\newcommand{\transf}{\theta}
%\newcommand{\transfset}{\Theta}

%e-cash protocol
%\newcommand{\adversary}{\mathcal{A}}
%\newcommand{\simulator}[1][]{\ifthenelse{\equal{#1}{}}{\mathcal{S}}{\mathcal{S}_{#1}}}
%\newcommand{\bank}{\mathfrak{B}}
%\newcommand{\user}[1][]{\ifthenelse{\equal{#1}{}}{\mathfrak{U}}{\mathfrak{U}_{#1}}}

%\newcommand{\transferlist}{\Pi}
%\newcommand{\info}{\mathfrak{info}}
%\newcommand{\wallet}{\mathfrak{wallet}}

%\newcommand{\coinname}{\mathfrak{coin}}
%\newcommand{\coin}[1][]{\ifthenelse{\equal{#1}{}}
%	{\coinname}
%	{\ifthenelse{\equal{#1}{0}}
%		{\coinname = (S, \pok_{S}, l, \pok_{L,l}, \pok_{\sigma}, \transferlist_T = \{T_0, \pok_{T_0}, r_0, \info_0\})}
%		{\coinname = (S, \pok_{S}, l, \pok_{L,l}, \pok_{\sigma}, \transferlist_T = \{ T_j, \pok_{T_j}, r_j, \info_j \}_{j=0 \ldots #1})}
%	}
%}
%\newcommand{\Coin}[1]{\coinname' = (S, \pok_{S}, l, \pok_{L,l}, \pok_{\sigma}, \transferlist_T = \{ T_j, \pok_{T_j}, r_j, \info_j \}_{j=0 \ldots #1})}

%\newcommand{\deposit}{\mathcal{CS}}

%tcg
%\newcommand{\player}[1][]{\ifthenelse{\equal{#1}{}}{\mathfrak{P}}{\mathfrak{P}_{#1}}}
%\newcommand{\server}{\mathfrak{G}}
%\newcommand{\register}{\mathfrak{C}}
%\newcommand{\market}{\mathfrak{M}}
%\newcommand{\auditor}{\mathfrak{A}}
%\newcommand{\report}{\mathcal{RS}}
%\newcommand{\culprit}{\mathcal{DS}}

%\newcommand{\CID}{{C\hspace{0.05em}I\hspace{-0.1em}D}}
%\newcommand{\UID}{{UI\hspace{-0.1em}D}}
%\newcommand{\validity}{V}
%\newcommand{\owner}{owner}

%\newcommand{\cardname}{\mathfrak{card}}
%\newcommand{\card}[1][]{\ifthenelse{\equal{#1}{}}
%	{\cardname}
%	{\ifthenelse{\equal{#1}{0}}
%		{\cardname = (\UID, \CID, \pok_{\UID}, \pok_{\sigma}, \transferlist_T = \{T_0, \pok_{T_0}, r_0, \info_0\})}
%		{\cardname = (\UID, \CID, \pok_{\UID}, \pok_{\sigma}, \transferlist_T = \{ T_j, \pok_{T_j}, r_j, \info_j \}_{j=0 \ldots #1})}
%	}
%}
%\newcommand{\Card}[1]{\cardname' = (\UID, \CID, \pok_{\UID}, \pok_{\sigma}, \transferlist_T = \{ T_j, \pok_{T_j}, r_j, \info_j \}_{j=0 \ldots #1})}

% Math -------------------------------------------------------------------
%\newtheorem{theorem}{Theorem}{\bfseries}{\itshape}
%\newtheorem{lemma}{Lemma}{\bfseries}{\itshape}
%\newtheorem{definition}{Definition}{\bfseries}{\itshape}
%\newtheorem{corollary}{Corollary}{\bfseries}{\itshape}
%\newtheoremstyle{example}{\topsep}{\topsep}%
%	{}%         Body font
%	{}%         Indent amount (empty = no indent, \parindent = para indent)
%	{\bfseries}% Thm head font
%	{:}%        Punctuation after thm head
%	{.5em}%     Space after thm head (\newline = linebreak)
%	{\thmname{#1}\thmnumber{ #2}\thmnote{ #3}}%         Thm head spec
%\theoremstyle{example}
%\newtheorem{example}{Example}


\renewcommand{\dissertacao}{%
  \renewcommand{\PoliTipoDocData}{Disserta\c{c}\~ao (Mestrado)}
  \comentario{\vspace{1.0cm}
            Disserta\c{c}\~ao apresentada \`a Escola Poli-
            t\'ecnica da Universidade de S\~ao Paulo 
            para a obten\c{c}\~ao do T\'itulo de Mestre em
            Engenharia de Computa\c{c}\~ao.}%
}

\renewcommand{\areaConcentracao}[2][\'Area de concentra\c{c}\~ao:\vspace{1mm}\\]%
  {\renewcommand{\ABNTareaconcname}{#1}%
   \renewcommand{\ABNTareaconcdata}{#2}}

\renewcommand{\orientador}[2][Orientador:\vspace{1mm}\\]%
  {\renewcommand{\ABNTorientadorname}{#1}%
   \renewcommand{\ABNTorientadordata}{#2}}

\sloppy

\begin{document}

\titulo{Dizang: Uma solução para coleta de evidências forenses para ataques de injeção na nuvem}

\autorPoli{Hamilton}{H.}{Fonte II}{Fonte}{II}
\orientador{Marcos Antonio Simplicio Junior}

\dissertacao{}
\areaConcentracao{Engenharia da Computa\c{c}\~ao}
\departamento{Departamento de Engenharia de Computa\c{c}\~ao e Sistemas Digitais (PCS)}

\local{S\~ao Paulo}
\data{2019}
\dedicatoria{A minha esposa e filha, pelas noites e finais de semana que não estive com vocês durante a elaboração deste trabalho}
\capa{}
\folhaderosto{}

%\begin{folhadeaprovacao}
% orientador
% prof dr banca 1
% prof dr banca 2
%\end{folhadeaprovacao}

\paginadedicatoria{}

%\fichacatalografica

\begin{agradecimentos}

As Deus por ter dado sua benção durante todo momento para que eu pudesse continuar
meus estudos.

A minha familia pelo apoio incondicional

Ao meu orientador o Prof. Dr. Marcos Simplício, que me amparou e conduziu nesta jornada.

A Universidade de São Paulo, a Escola Politécnica e o Programa de Pós Graduação,
por ter me acolhido como aluno de mestrado.

\end{agradecimentos}


\begin{resumo}
A adoção de arquiteturas em nuvem aumenta a cada dia, e proporcionalmente aumenta também o número de casos em que esse tipo de tecnologia é usada para fins ilícitos.
%
%No ano de 2015 o número de registros de atas notariais comprovando abusos e crimes virtuais cresceu 87\%. Muito específico para um resumo
%
Infelizmente, devido à natureza volátil da nuvem, a tarefa de coletar evidências para análise forense nesse ambiente tem esbarrado em desafios práticos e legais.
%
Mais precisamente, a prática herdada da forense tradicional para coleta de evidências, por meio da qual se isola a cena do crime e coletam-se todas as evidências, foi traduzida para a forense digital como a cópia bit a bit da mídia que se deseja investigar.
%
Tal prática leva à coleta de grandes volumes de informação para análise, impactando negativamente o tempo de investigação.
%
Além disso, duas características das soluções em nuvem dificultam a obtenção de evidências válidas. 
%
A primeira é que o compartilhamento de recursos físicos entre vários usuários impede a sua remoção para análise, uma vez que isso violaria a privacidade de indivíduos não envolvidos na investigação.
%
A segunda é que a localização do recurso físico em uma região geográfica diferente daquela onde o crime foi cometido pode impedir a coleta de evidências caso não haja acordos de cooperação estabelecidos.
%
Estes aspectos, se não forem levados em consideração, podem colocar em xeque a credibilidade das evidências e diminuir a chance de serem aceitas em um processo legal. %colocando aceitabilidade no texto
%
Este trabalho, com foco técnico, analisa propostas na literatura voltadas a resolver tais desafios na coleta evidências na nuvem, discutindo suas limitações.
%
Propõe-se então uma solução que cobre coleta, transporte e armazenamento da evidência, visando suplantar as limitações existentes no estado da arte. 
%
A solução proposta, denominada \fancyname, provê uma forma de correlacionar evidências e sua origem virtual, permitindo transportar e armazenar tais dados sem afetar sua credibilidade.
%
Para tal, é proposto um mecanismo de identificação única do gerador da evidência, técnica esta que tem a vantagem que conseguir preservar a relação evidência-origem mesmo que esta última não exista mais na solução sob investigação. %tornando mais genérico, removendo contêiner
%
Em resumo, \fancyname tem como focos (1) a reprodutibilidade do processo de coleta, (2) o estabelecimento de um vínculo entre a evidência coletada e sua origem, (3) a preservação da jurisdição e da privacidade de usuários não envolvidos na investigação e (4) a garantia de custódia da evidência.
%
Atingindo estes objetivos, \fancyname contribuirá também para o processo de gestão de incidentes.
\end{resumo}

\begin{abstract}
%\marcos{Reescreva considerando as revisões na parte em português - Hamilton: Feito}
The adoption of cloud architectures increases every day, and with it increases the number of cases in which this type of technology is used for illicit purposes.
%
Unfortunately, due to the volatile nature of the cloud, the task of collecting evidences for forensic analysis in this environment has run into practical and legal challenges.
%
The practice inherited from the traditional forensics, through which the crime scene is isolated and all the evidence is collected, when applied to digital forensics, led to the collection of large volumes of information for analysis.
%
Such practice had a negative impact on the time of investigation. 
%
In addition, two characteristics of cloud solutions make it difficult to obtain valid evidence. 
%
The first is that the physical resources are shared among multiple users which prevents their removal for analysis, as this would violate the privacy of individuals not involved in the investigation.
%
The second is the resource's physical location which can be in a different geographical region where the crime was committed and may prevent the collection of evidence if there are no cooperation agreements in place. 
%
These aspects, if not taken into consideration may call into question the credibility of the evidence and hamper its acceptance in a court of law. 
%
This work, technicaly driven, analyzes proposals in the literature aimed at solving such challenges, discussing its limitations and proposes a solution that covers collection, transportation and storage of evidence, aiming at overcoming existing limitations in the state of art.
%
The proposed solution, called Dizang, provides a way of correlating evidence and its virtual origin, allowing the transport and storage of such data without affecting its credibility. 
%
For this, it proposes a mechanism to identify uniquely the source of the evidence, in order to preserve the relationship evidence-source even though the latter does not exist in the system under investigation.
%
In summary, Dizang focuses on (1) the reproducibility of the collection process, (2)the establishment of a link between the evidence collected and its origin, (3) the preservation of jurisdiction and privacy of users not involved in the investigation and (4)the evidence's chain of custody.
%
In reachnig those objectives, \fancyname will contribute to incident handling processes too.
\end{abstract}

\listoffigures
\listoftables

\begin{listofabbrv}{FlexDHE}
    \item [VM] \textit{Virtual Machines}
    \item [SO] Sistema Operacional
    \item [SaaS] \textit{Software as a Service}
    \item [PaaS] \textit{Platform as a Service}
    \item [IaaS] \textit{Infrastructure as a Service}
    \item [FaaS] \textit{Forensics as a Serviço}
    \item [LXC] \textit{Linux Conteiners}
    \item [VMI] \textit{Virtual Machine Introspection}
    \item [VMM] \textit{Virtual Machine Manager}
    \item [CSP] \textit{Cloud Service Provider}
    \item [GRR] \textit{Google Rapid Response}
    \item [FROST] \textit{FoRensic Open Stack Tools} 
    \item [API] \textit{Aplication Programing Interface}
    \item [SWGDE] \textit{Scientific Working Group on Digital Evidence}
    \item [ETL] \textit{Extract, Transform and Load}
    \item [EDIPM] \textit{Enhanced Digital Investigation Process Model}
    \item [SENASP] Secretaria Nacional de Segurança Pública
    \item [ETL] \textit{Extract, Transform and Load}
    \item [NAS] \textit{Network Accessed Storage}
    \item [GDB] \textit{GNU Debugger}
\end{listofabbrv}

%\begin{listofsymbols}{1000000}
%\item[$a>>b$] Símbolo 1
%\item[$||$] Símbolo 2
%\item[$|x|$] Símbolo 3
%\item[$\oplus$] Símbolo 4
%item[$gcd(a,b)$] Símbolo 5
%\item[$\hash$] Símbolo 6
%\item[$\bigO$] Símbolo 7
%\end{listofsymbols}


\tableofcontents

% ----------------------------------------------------------
% Introdução
% ----------------------------------------------------------
\chapter{Introdução}

\section{Problema de pesquisa}

Falar do problema forense em nuvem

\section{Objetivos}

Com resolveremos este problema

\section{Justificatica}

Porque isto se justifica

\section{Método de pesquisa}

Como eu vou chegar a este resultado

\section{Organização de documento}

O bla bla bla de sempre
% ----------------------------------------------------------
% Capitulo de Fundamentação teórica 
% ----------------------------------------------------------
\chapter{Fundamentação Teórica}
\label{chp:fundamentação}


Desde 2001, diversos modelos para a condução de investigação digital foram propostos. 
%
No \textit{First Digital Forensic Research Workshop} foi definido o primeiro modelo de processo investigativo genérico para atender às investigações envolvendo sistemas digitais.
%
Em seguida foi proposto o \textit{Abstract Digital Forensic Model}, mais detalhado que o modelo genérico anterior e o \textit{Integrated Digital Investigation Process}, baseado em teoria criminal e investigações não digitais do passado.
%
Finalmente em 2003, Carrier e Stafford propõem a última iteração na evolução do processo forense digital, o \textit{Enhanced Digital Investigation Process Model} \cite{SimouCloudChlng:2014}.
%
Entretanto, como tais modelos de investigação foram desenvolvidos antes da aparição de tecnologias de computação em nuvem, muitos partem do pressuposto que o investigador tem acesso e controle sobre o sistema sob investigação \cite{GrisposChallengesCloudComputing:2012}.
%
Novas pesquisas são necessárias para que a forense digital aborde apropriadamente as soluções em nuvem e suas particularidades.


Neste capítulo, são discutidos os principais conceitos que permeiam esse cenário.
%
Especificamente, após uma discussão geral sobre computação em nuvem e suas particularidades, são discutidas os características esperadas de um processo de forense digital robusto e sua interseção com gestão de incidentes.


\section{Nuvens computacionais e contêineres}
\label{sec:computacaonuvem}

Uma nuvem computacional é um modelo de infraestrutura no qual recursos compartilhados em quantidade configurável, acessíveis via rede, são alocados e desalocados com esforço mínimo de gerenciamento por parte de um provedor de serviços.
%
Existem três modelos principais de comercialização de uso da nuvem \cite{NIST2011}: 

\begin{itemize}
	\item \textit{Software} como serviço ( \textit{Software as a Service} -- SaaS ): No qual se provê o \textit{software} que será utilizado; nesse caso, os clientes do serviço são os usuários finais do software.
	
	\item Plataforma como serviço ( \textit{Platform as a Service} -- PaaS ): No qual se provê o ambiente para o desenvolvimento, teste e execução do \textit{software}; nesse caso, os clientes do serviço são desenvolvedores de aplicações.
	
	\item Infraestrutura como serviço ( \textit{Infrastructure as a Service} -- IaaS ): No qual são fornecidos recursos computacionais básicos, como processamento, memória e redes, em geral de forma virtualizada; os clientes desse tipo de serviço costumam ser arquitetos de sistemas.
\end{itemize}

O tipo de serviço de nuvem mais pertinente para este trabalho é o IaaS, uma solução muito usada atualmente pela sua capacidade de prover recursos sob demanda de forma auto-escalável.
%
Nesse cenário, o uso intenso de tecnologias de virtualização costuma levar a recursos altamente voláteis, que são alocados e desalocados a qualquer momento pelo orquestrador da nuvem para suprir eventuais aumentos e reduções de demanda.
%
É possível até mesmo construir \textit{scripts} para a automatizar a construção da arquitetura desejada, permitindo a instanciação e interconexão de máquinas adequadas para a atividade fim do sistema.
%
Esses \textit{scripts}, escritos em linguagem específica como Puppet \cite{Puppet2018}, Chef \cite{Chef2018} ou Vagrant \cite{Vagrant2018}, podem conter instruções de como distribuir o tráfego de rede entre as diferentes instâncias de computação ou armazenamento.
%
%Dentre as vantagens de sua utilização, podem ser citadas a capacidade de usar os recursos de nuvem de uma forma mais eficiente, levar a uma menor necessidade de intervenção humana, e prover maior resiliência a variações de demanda do sistema.


%\section{Uso de contêineres}
%\label{sec:conteiner}

Uma tecnologia de virtualização possível para cenários de nuvem, e cuja utilização vem crescendo nos últimos anos, são os chamados contêineres \cite{containers-tech:2014}. 
%
Basicamente, um contêiner é um método de virtualização do sistema operacional que permite executar uma aplicação, bem como suas dependências, em um processo no qual recursos como disco, memória e rede permanecem isolados.
%
Diferente das máquinas virtuais, a virtualização com contêineres é feita no nível do sistema operacional (SO) nativo.
%
Como resultado, tem-se uma implementação de virtualização na qual eliminam-se camadas entre o aplicativo executado e o \textit{hardware} físico, permitindo maior granularidade no controle sobre esses recursos e melhorando a eficiência da infraestrutura.


Uma implementação bastante utilizada para esse propósito são os Contêineres Linux ( \textit{LinuX Conteiners} -- LXC ) \cite{Linuxcontainers.org2015}, que aproveitam-se de funcionalidades como \textit{cgroups}, \textit{kernel namespacing} e \textit{chroot} do núcleo do Linux para auxiliar no gerenciamento e isolamento de recursos virtuais.
%
Mais precisamente, a funcionalidade de \textit{cgroups} ( \textit{Control Groups} -- Grupos de Controle ) presente no núcleo do Linux limita e isola o uso de recursos como CPU, memória e disco de um conjunto de processos, além de organizá-los de forma hierárquica. 
%
O trabalho nessa funcionalidade começou em 2006 na Google sob a denominação de \textit{process container}. 
%
No final de 2007, seu nome foi alterado para \textit{control groups}, e o resultado foi então adicionado à versão 2.6.24 do núcleo lançado em 2008 \cite{UnixManPagesControlGroups}.

Já o \textit{Namespacing} é uma funcionalidade do núcleo do Linux usada para isolar e virtualizar recursos do sistema operacional, como identificadores de processos, acessos à rede, comunicação inter-processos e sistema de arquivos.
%
\textit{Namespacing} envolve os recursos do sistema operacional em uma abstração que faz parecer aos processos de um mesmo \textit{namespace} que estes tem sua própria instância isolada de um recurso global.
%
Desta forma, essa é a principal funcionalidade por trás da implementação de Contêineres Linux \cite{UnixManPagesNamespacing}.


Finalmente, \textit{chroot} ( \textit{Change Root} -- Troque a raiz do sistema de arquivos ) é uma funcionalidade do núcleo do Linux usada para mudar o diretório \textit{root} utilizado pelo processo que está chamando a função, bem como por todos os seus processos filhos. 
%
A chamada a \textit{chroot} altera o processo de resolução de caminhos do sistema operacional para o processo que o chamou \cite{UnixManPagesChRoot}.
%
Desta forma, pode-se instalar uma distribuição Linux secundária em uma pasta, ao invés de uma partição, e executar programas desta pasta sem perda significativa de desempenho.


\section{Forense digital e seus desafios}
\label{sec:forensedigital}


A área de forense digital (também conhecida por forense computacional) refere-se a um conjunto de técnicas de coleta e análise da interação entre humanos e computadores de forma que suas conclusões sejam aceitas em um processo legal.
%
Tal como a forense tradicional, a forense digital se baseia no princípio de Locard, estabelecido pelo médico francês Edmond Locard da seguinte forma: ``Quando um indivíduo entra em contato com outro objeto ou indivíduo, este sempre deixa vestígio deste contato'' \cite[p.~31]{Ramos:2011}.
%
De forma similar, a forense digital tem por objetivo a investigação de evidências digitais da interação entre homem e máquina, de modo a reconstruir a cadeia de eventos passados para que suas conclusões sejam validadas por terceiros e sejam aceitas em um processo legal.
 

Na próxima subseção são detalhados os principais desafios enfrentados pela forense digital quando aplicada a infraestruturas em nuvem.

\subsection{Admissibilidade da evidência em processo legal.}
\label{sec:credibilidadeaceitabilidadeevidencia}

O processo de análise forense no evento de um crime digital é descrito no Modelo Melhorado para Processo de Investigação Digital (\textit{Enhanced Digital Investigation Process Model} -- EDIPM) na forma de 4 fases \cite{GrisposChallengesCloudComputing:2012}: identificar, preservar, examinar e apresentar.
%
A fase mais pertinente a este trabalho é a de preservação da evidência, que deve ser conduzida de forma forensicamente aceitável.
%
Ou seja, deve-se coletar as evidências de forma que estas sejam aceitas em um processo legal e não tenham sua credibilidade questionada no curso do mesmo.


Para atingir tal objetivo, o primeiro passo é a garantia da cadeia de custódia relacionada a evidência.
%
Cadeia de custódia é o processo de documentação da história cronológica da evidência de modo a saber onde a evidência esteve e quem teve acesso a esta \cite[p.~21]{Ramos:2011}. 
%
Uma cadeia de custódia idealmente deve tornar visível o estado da evidência antes, durante e após a interação com o processo de investigação \cite{LuisDigitalChainOfCustody:2016}.
%
A razão para essa preocupação é que alguns processos investigativos podem causar alteração da evidência na forma que foi coletada, como, por exemplo, a consolidação em um único local dos arquivos que compõem uma base de dados distribuída.
%A SENASP (Secretaria Nacional de Segurança Pública) \marcos{Que sigla é essa? Cadê a referência? - Hamilton: removido, repetitivo} define cadeia de custódia como ``a sistemática de procedimentos que visa à preservação do valor probatório da prova pericial caracterizada.''


O passo seguinte é a garantia da autenticidade e da integridade da evidência.
%
Autenticidade pode ser definida como ``o processo pelo qual se pode garantir a autoria do documento eletrônico'' \cite{Ramos:2011}, ou seja, por meio do qual se não permite dúvida quanto à identificação do autor .
%
Já a integridade pode ser definida como ``o atestado da inteireza do documento eletrônico após sua transmissão, bem como apontar eventual alteração irregular de seu conteúdo'' \cite{Ramos:2011}. 
%
Caso haja dúvida acerca de um desses requisitos, uma perícia técnica pode ser convocada.
%
Nesse caso, durante a perícia é analisado o autor da evidência, ou seja, verifica-se sua fonte e se a mesma não foi alterada no processo.


Em uma infraestrutura física, a coleta de evidências pode ser feita de forma relativamente simples, bastando-se remover o recurso físico, transportar este para um laboratório e lá analisar os dados. 
%
Para limitar a exposição da evidência a manipulações indevidas, esta pode ser mantida em uma sala-cofre, à qual o acesso é controlado.
%
A reprodutibilidade do processo de coleta e a manutenção da integridade da evidência são, então, tarefas bem diretas.


Já em um cenário de computação em nuvem, especialmente as de infraestrutura auto-escalável, existe um conjunto de novos desafios. 
%
Primeiramente, em contraste com infraestruturas tradicionais, o recurso físico em princípio não pode ser removido: como os recursos são utilizados por outros usuários não relacionados à investigação, fazê-lo constituiria violação de privacidade.
%
A volatilidade dos recursos também torna a verificação do seu autor um processo mais complexo, pois o recurso que gera certa evidência pode deixar de existir algum tempo depois de fazê-lo \cite{SimouCloudChlng:2014}.
%
A integridade da evidência também acaba sendo uma tarefa não trivial, pois ela precisa ser coletada, transportada e armazenada, o que caracteriza a necessidade de preservação da cadeia de custódia.
%
A violação de qualquer uma dessas características pode colocar em dúvida a credibilidade da evidência. Embora a admissibilidade da evidência em um processo legal seja uma decisão do juiz a credibilidade da mesma tem papel importante nesta decisão.
%Infelizmente, a violação de qualquer uma dessas características pode colocar em dúvida a credibilidade da evidência.


\subsection{Volume de dados para coleta}
\label{sec:volumedados}

O processo de coleta de evidências na forense digital herda suas práticas da forense tradicional, na qual isola-se cena do crime e coletam-se as evidências presentes. 
%
Transportando esse método para a forense digital, introduz-se a realização da cópia bit a bit da informação que se deseja analisar.
%
No passado, com as soluções manipulando quantidades bem menores de memória, disco e tráfego, tal prática não era considerada muito problemática. 
%
Entretanto, nas atuais soluções, aplicações e arquiteturas em nuvem, o volume de dados é consideravelmente maior \cite{QuickIncreaseVolumeImpact:2014}.
%
Afinal, o acesso fácil e dinâmico a recursos de armazenamento e de processamento, como máquinas virtuais, balanceadores de carga e \textit{firewalls}, acaba também aumentando a quantidade elementos geradores de evidência.
%
Esse problema é ilustrado em \cite{QuickIncreaseVolumeImpact:2014}, na qual se discute uma investigação na qual a quantidade de dados a serem analisados para fins de forense digital tomaria 6 meses.
%
Encontrar uma forma de armazenar menos informações e, assim, tornar a fase de análise mais rápida e eficiente, é um passo importante para garantir a celeridade de investigações forenses.


\subsection{Privacidade e jurisdição}
\label{sec:violacaoprivacidadejuriscdicao}

%No método tradicional de coleta de evidências para análise, isola-se o ambiente e as evidencias são removidas. \marcos{Eu acho que já li essa frase...}
%
%Transportando para a forense digital, temos a prática de remover o equipamento para realização de cópia bit a bit da evidência. 
%
%\marcos{Eu acho que já li essa frase... (sim, está repetitivo... resuma em uma frase curta, potencialmente fazendo referência à seção anterior) - Hamilton: Assim?}
%
A prática de remoção de equipamentos para coleta de evidências, mencionada na subseção \ref{sec:volumedados}, também traz consequências negativas do ponto de vista de privacidade.
%
Mais precisamente, em soluções envolvendo infraestruturas físicas, tal prática não costuma trazer grandes problemas porque os objetos ou indivíduos sob investigação normalmente estão diretamente relacionados ao equipamento removido.
%
Nas soluções em nuvem, entretanto, tal prática não é recomendada porque o recurso físico é compartilhado por vários usuários, inclusive indivíduos não envolvidos na investigação.
%
Logo, remover tais recursos configuraria violação de privacidade.
%
Como um complicador adicional, o fato de os dados potencialmente não estarem armazenados no mesmo território em que a investigação é realizada acaba demandando acordos de cooperação jurídica entre as partes, o que nem sempre é possível \cite{SimouCloudChlng:2014}.


O ''State of the Cloud Report'', relatório produzido pela empresa Right Scale \cite{RightScale2018}, cita nas páginas 21 e 22 que os maiores desafios na adoção de nuvem são, \textit{gerenciamento de custo} e \textit{segurança}.
%
Embora privacidade e jurisdição sejam aspectos distintos e possuidores de suas particularidades, neste documento estes são discutidos em conjunto pois em busca de otimização de custos as empresas tendem a buscar regiões mais baratas e compartilhamento do recurso físico.
%
Neste cenário, encontrar uma forma de coletar a evidência sem violar jurisdição e privacidade ganham importância.


\subsection{Coleta de evidências de memória volátil de máquinas em nuvem}
\label{sec:forensenuvem}

A prática de armazenar histórico de tráfego de rede e alterações de dados armazenados em disco já é bem difundida na comunidade forense.
%
Por outro lado, a memória volátil de computadores não costuma receber o mesmo tratamento: suas alterações quase nunca são armazenadas, seja por questões de desempenho ou por simples praticidade, dado a reduzida sobrevida dessas informações.
%
Infelizmente, isso acaba dificultando a análise de uma classe específica de ataques, conhecidos como injeção de código em memória \cite{CaseMemoryForensics:2014}. 
%
Assim, quando usados contra uma arquitetura em nuvem, tais ataques não deixam rastros quando recursos de processamento virtuais são desativados e sua memória é liberada \cite{VomelMemoryAcquisition:2013,CaseMemoryForensics:2014}.
%
Em particular, têm especial interesse quatro tipos particulares dessa família de ameaças \cite{CaseMemoryForensics:2014}:


\begin{itemize}
 \item \textbf{Injeção remota de bibliotecas}: Um processo malicioso força o processo alvo a carregar uma biblioteca em seu espaço de memória.
 %
 Como resultado, o código da biblioteca carregada executa com os mesmos privilégios do executável em que ela foi injetada. 
 %
 Tal estratégia, comumente usada para instalar códigos maliciosos, pode fazer com que uma biblioteca maliciosa armazenada no sistema seja distribuída por vários processos de uma mesma máquina, dificultando sua remoção \cite{MillerRemoteLibraryInjection:2004}.
 
 \item \textbf{Inline Hooking}: Um processo malicioso escreve código como uma sequência de bytes diretamente no espaço de memória de um processo alvo, e então força este último a executar o código injetado. 
 %
 O código pode ser, por exemplo, um \textit{shell script}.
 

 \item \textbf{Injeção reflexiva de biblioteca}: Um processo malicioso acessa diretamente a memória do processo alvo, inserindo nela o código de uma biblioteca na forma de uma sequência de bytes, e então força o processo a executar essa biblioteca. 
 %
 Nessa forma de ataque, a biblioteca maliciosa não existe fisicamente; isso torna tal estratégia de injeção de código potencialmente mais atrativa, pois o carregamento da biblioteca não é registrado no sistema operacional (SO), dificultando a detecção do ataque \cite{FewerReflectiveLibraryInject:2008}.
 
 \item \textbf{Injeção de processo vazio}: Um processo malicioso dispara uma instância de um processo legítimo no estado ``suspenso''; a área do executável é então liberada e realocada com código malicioso.
\end{itemize}


\section{Forense digital e gestão de incidentes}
\label{sec:forenseeincidentes}

%
Gestão de incidentes é uma estratégia para reduzir os riscos à confidencialidade, integridade e disponibilidade de ativos de uma organização, bem como minimizar sua perda.
%
As fases relevantes de uma estratégia de gestão de incidentes são \cite{AdhiantoFasesGestaoIncidente:2010}: 


\begin{itemize}
%
\item \textbf{Preparação}: Onde se organiza o ambiente para minimizar impactos de um incidente, nesta fase também é definido o time de resposta a incidente que tem como responsabilidade determinar o que ocorreu e quais ações devem ser tomadas.
%
\item \textbf{Detecção}: Tem como objetivo minimizar o risco das ameaças que não foram antecipadas. A fase de detecção começa quando alguma atividade suspeita é detectada por algum serviço automático como um sistema de detecção de ameaças ou por intervenção humana como o de um responsável pela análise de risco.
%
\item \textbf{Resposta ao incidente}: Contém as fases de \textit{contenção}, \textit{erradicação} e \textit{recuperação}. Não existe um procedimento que atenda a todas os cenários, os procedimentos são feitos de acordo com a natureza e as necessidades do negócio.
%
\item \textbf{Pós incidente}: Última fase da gestão de incidentes, sua principal preocupação é coleta de informações das três fases anteriores para alimentar um processo de aprendizado visando evitar futuros incidentes semelhantes.

\end{itemize}

A gestão de incidentes tem o foco em conter falhas de segurança. 
%
Considerações como coleta de evidências costuma ser secundário \cite{AdhiantoFasesGestaoIncidente:2010}. 
%
A oportunidade de se aplicar processos de coleta forenses em gestão de incidente já foi levantado por \cite{AdhiantoIncidentHandlingForensic:2016} uma vez que ambos compartilham de um ferramental em comum.
%

%
Ferramental e técnicas forenses estão se tornando úteis não só para geração de evidências para processos jurídicos mas também para ajudar na reconstrução de eventos.
%
Incorporar práticas forenses na estratégia de gestão de incidente, ajuda na geração das evidências necessárias para eventual processo jurídico como também ajuda a proteger as evidencias de dano como resultado das fase de resposta ao incidente.






% ----------------------------------------------------------
% Material e métodos
% ----------------------------------------------------------
\chapter{Revisão Bibliográfica}

Aqui vai a revisão Bibliográfica

% ----------------------------------------------------------
% Proposta
% ----------------------------------------------------------
\chapter{Proposta de projeto: \fancyname}
\label{chp:proposta}

A presente proposta tem como objetivo principal coletar memória de recursos computacionais virtuais em arquitetura volátil de modo a conseguir: 
(1) identificar a fonte da evidência, mesmo se o recurso virtual não existir mais; 
(2) descrever o sistema antes e depois do incidente;
(3) transportar e armazenar a memória coletada de uma forma que garanta sua integridade e confidencialidade; e
(4) não violar a jurisdição e a privacidade de outros usuários que porventura tenham recursos alocados no mesmo servidor físico.
%
A solução aqui apresentada, denominada \fancyname, é descrita em detalhes a seguir.

\section{Identificação da origem}
\label{sec:proposal-desc-origin}

Em sistemas computacionais executados sobre uma infraestrutura física (i.e., não virtualizada), pode-se fazer uma associação direta entre um recurso qualquer e sua origem correspondente, seja este recurso uma informação da memória, imagem de disco ou pacotes trafegando na rede.
%
Já em sistemas construídos sobre uma infraestrutura virtual, em especial quando ela é auto-escalável, os recursos computacionais são altamente voláteis e, portanto, podem ser desalocados a qualquer momento.
%
Este fato torna difícil a associação de uma informação gerada por esta infraestrutura com sua origem.


Para conseguir correlacionar uma evidência a sua origem volátil, é necessário utilizar outro elemento em que persista a relação fonte-evidência.
%
O presente trabalho propõe que isto seja feito por meio de cálculo de hash do recurso em nuvem que produziu a evidência. %removendo contêiner para deixar mais genérico
%
%Embora um contêiner seja um \textit{software} e, portanto, também volátil, cada imagem compilada e sua execução na forma de contêiner são normalmente atrelados a um \textit{hash} que identifica univocamente essa relação. %removendo para deixar mais genérico
%
O hash de um recurso em nuvem permite identificar univocamente a fonte de uma evidência. Em arquiteturas que utilizam contêiner por exemplo, é possível identificar se a evidência veio do contêiner do motor de páginas dinâmicas (e.g., Apache%\cite{Tomcat}
), do contêiner da lógica de negócios (e.g., \textit{golang}%\cite{Google}
) ou do contêiner do banco de dados (e.g., \textit{Cassandra}%\cite{Cassandra}
). %removendo contêiner e deixando mais genérico

\section{Descrever o sistema antes e depois do incidente}
\label{sec:proposal-desc-incident}

%\marcosT{O parágrafo estava muito grande, então quebrei aqui. O problema é que falta uma frase para ligar a frase a seguir ao contexto da discussão... Coloque uma frase aqui, deixando claro qual requisito você quer satisfazer com essa ideia de ``interromper temporariamente a execução do contêiner''. Aplique isso para TUDO que for proposta: o leitor tem que saber de antemão pra que você está fazendo alguma coisa, ou vai ficar se perguntando ``Espera, mas pra que fazer isso?!'' - Hamilton: Feito}
A cópia de memória não é uma atividade atômica, pois ela é executada em conjunto com outros processos. 
%
Portanto, caso um desses processos seja um código malicioso apagando traços de sua existência da memória do recurso, informações possivelmente importantes para a investigação podem acabar sendo perdidas. 
%
Com o objetivo de deixar o processo de cópia da memória mais atômico, a fim de evitar inconsistências na informação coletada \cite{CaseMemoryForensics:2014}, \fancyname propõe que a execução do recurso em nuvem seja temporariamente suspenso para que seja realizada a cópia de sua memória. 
%
Essa técnica, que é semelhante àquela adotada em \cite{RafiqueStaticLiveDigitalForensics:2013} para VMs, produz um instantâneo da memória volátil do recurso; isso permite sua análise em um estado de repouso, ou seja, sem a necessidade de ter o recurso em execução.
%
Ao realizar a coleta em intervalos de tempo adequados, é possível construir um histórico do estado da memória durante a execução no recurso.
%
%\marcosT{A ligação entre as frases anterior e a seguir está bem ruim... você parece estar mudando completamente de assunto... Acredito que faltou uma frase dizendo que ``''salvar toda a memória'' pode se tornar um problema, reforçar isso com as duas frases que já estão a seguir, e depois dizer que você vai resolver.}


A maioria das técnicas forenses mais usadas atualmente são voltadas à obtenção da informação em sua totalidade.
%
Isso comumente é feito via cópia bit a bit ou por meio da obtenção do \textit{hardware} físico \cite{SimouCloudChlng:2014} \cite{BemPastPresentFuture:2008}. 
%
Embora tais técnicas possam parecer interessantes à primeira vista, elas muitas vezes acabam sendo responsáveis por um problema: o crescente volume de informações que os investigadores precisam analisar \cite{QuickIncreaseVolumeImpact:2014}.
%
Para mitigar essa dificuldade, em \fancyname são adotadas duas estratégias: a primeira é a definição de um volume de dados que possa ser considerado \textit{suficiente} para a realização de uma investigação; a segunda é a definição de uma \textit{idade máxima} para a evidência enquanto o sistema trabalha em condições normais, isto é, quando não está sob ataque.
%
Para detectar e analisar intrusões na memória de processos, é necessário ter uma cópia da memória antes e depois da intrusão \cite{CaseMemoryForensics:2014}. 
%
Assim, a solução proposta implementa uma janela de instantâneos de memória cobrindo um intervalo de tempo pré-definido, como ilustrado na Figura \ref{fig:janela}. 
%
Em condições normais de operação, as evidências são coletadas com certa periodicidade e coletas que atingem uma determinada idade são descartadas.
%
Em contraste, após a detecção de um evento de ataque (e.g., por um sistema de detecção de intrusões), \fancyname deixa de descartar as coletas mais antigas do \textit{log} de monitoramento.
%
Como resultado, é possível conhecer o sistema antes e depois do ataque e, assim, avaliar sua evolução.
%
%\marcosR{Não sei por que você está utilizando $\backslash\backslash$ no final das suas frases, mas pare de fazer isso... deixe o LaTeX se virar com a formatação. No final, quando você tiver tudo escrito, aí pode fazer algum sentido se preocupar com formatação, mas não antes disso... - Hamilton: é pra formatação mesmo :-) OK vou deixar o latex se virar}
%

\begin{figure}[htb!]
\footnotesize
\caption{Janela deslizante de coleta de evidência}
\includegraphics[scale=1.00]{janela.pdf}
\centering
\label{fig:janela}
\begin{center}
Fonte: Próprio autor 
\end{center}
\end{figure}

\section{Garantindo integridade, confidencialidade e protegendo privacidade e jurisdição}
\label{sec:proposal-desc-chain-of-custody}

%\marcosT{Essa frase não faz sentido: não se ``assina'' nada com um ``hash''. Você pode ``calcular o hash'' ou ``assinar um dado'' (e.g., um dado juntamente com o hash de alguma coisa). Revise essa frase... - Hamilton: Feito}
Finalmente, para persistir a relação evidência-origem e garantir a sua integridade, \fancyname calcula o hash $H$ do par \{evidência, identificador da imagem do contêiner\} e armazena a tripla \{$H$, identificador do recurso, evidência\}.
%
Adicionalmente, a presente proposta evita eventuais problemas com o armazenamento desses dados em países com jurisdições diferentes daquelas que devem ser aplicadas na investigação em questão.
%
Especificamente, as evidências coletadas são armazenadas em um local físico fora da nuvem, após serem transportadas por meio de um canal seguro (e.g., via TLS (\textit{Transport Layer Security} -- Camada de Transporte Seguro) \cite{DierksT2008}).
%

\section{Implementação}
\label{sec:proposta-impl}

%\marcosT{CLAREZA: quais ataques? Onde estão esses objetivos (diga a seção!!!)? Perceba que não tem qualquer seção com esse nome: você espera realmente que o leitor procure no seu texto onde eles estão...? - Hamilton: Feito}

\begin{figure}[htb!]
\footnotesize
\caption{Arquitetura geral da solução Dizang}
\includegraphics[scale=0.70]{Solucao.pdf}
\centering
\label{fig:Solucao}
\begin{center}
Fonte: Próprio autor 
\end{center}
\end{figure}

%
Os mecanismos propostos foram implementados em uma plataforma de testes visando avaliar a eficácia de \fancyname em coletar as informações de memória dos contêineres de forma reprodutível, sem violar jurisdições ou a privacidade de usuários e a capacidade de detectar injeção de código usando as evidências coletadas.
%
A solução, ilustrada na Figura \ref{fig:Solucao}, consistiu na criação de uma instancia \textit{t2.micro} na zona Ohio da AWS com 3.3Mhz, 1Gb de RAM e sistema operacional de 64 bits. 
%na criação de 1 VM usando o Oracle Virtual Box 5.0%\cite{VirtualBox} em um notebook Intel i5 de 2.30Mhz e 4Gb de RAM com sistema operacional de 64 bits.
%
Nesta instância AWS foi instalado o Docker Engine 1.10 e a API Docker 1.21, com os quais foram criados 3 contêineres executando o nginx 1.0 em diferentes portas. 
%
Foi desenvolvida uma aplicação Java que, executada no sistema operacional hospedeiro, descobre o identificador de processo associado a cada contêiner, copia o conteúdo do \textit{descritor de alocação de memória não uniforme} (\textbf{/proc/pid/numa\_maps}), o qual contém a alocação das páginas de memória, os nós que estão associados a essas páginas, o que está alocado e suas respectivas políticas de acesso \cite{UnixManPagesNumaMaps}.
%
A cópia e gravação do arquivo é tal que, a cada intervalo de tempo $t$, a aplicação (1) pausa o contêiner em questão, (2) copia a diretório \textbf{numa\_maps}, (3)  concatena os dados obtidos com o identificador da imagem e do contêiner, (4) calcula o $H$ do conjunto e (5) salva o resultado em um arquivo cujo nome é o identificador da imagem e do contêiner e a extensão é \textbf{.mem}. 
%
O transporte seguro da evidência para um armazenamento físico fora da AWS foi implementado usando uma instância \textit{t2.micro} na zona Ohio da AWS onde foi instalado um servidor \textit{OpenVPN}.
%
Como uma forma básica de controle de acesso, a instância EC2 que contém as evidências foi configurada para aceitar conexões apenas de máquinas nesta VPN.
%
Uma máquina física fora da AWS, usou o cliente do \textit{OpenVPN} para estabelecer uma conexão VPN com a instância que contém as evidências e as transportou para o disco da máquina física.
%
Após a conclusão do processo de transporte, a máquina física verifica se existem arquivos \textbf{.mem} em disco mais antigos que um certo intervalo de tempo $t$, descartando-os.
%


\section{Resultados experimentais}
\label{sec:proposta-exp}

Para avaliar a efetividade de \fancyname na coleta de evidências e identificação de injeção de código, dois experimentos foram realizados usando o ambiente implementado (descrito na Seção \ref{sec:proposta-impl}).
%


\subsection{Análise do desempenho}
\label{sec:proposta-exp-desempenho}

No primeiro experimento, o sistema foi configurado para realizar coletas de memória em intervalos de 1 minuto, salvá-las em armazenamento externo à nuvem e apagar amostras coletadas há mais de 5 minutos. 
%
O sistema foi então executado por 30 minutos, tempo durante o qual foram coletadas como métricas (1) o uso de espaço em disco utilizado pelos instantâneos de memória salvos, (2) o tempo de pausa no contêiner necessário para a cópia delas e (3) o tempo de transporte das evidências para o armazenamento externo a nuvem.


A evolução do espaço em disco ocupado pelos instantâneos de memória, acompanhado através da execução do comando \texttt{du -sh *.mem} do \textit{Unix} no disco de armazenamento externo, é mostrada no gráfico da Figura \ref{fig:evolucao-coleta}.
%
Neste experimento os instantâneos de memória tem 244kb de tamanho. 
%
O gráfico mostra que o aumento do uso do espaço em disco é linear e o crescimento se interrompe quando é atingido o limite de tempo configurado para a janela, pois as coletas com tempo de vida maior que tal limite são apagadas do disco. 
%
Assim, a solução mantém sob controle o espaço em disco ocupado pelas amostras coletadas.
%
Ao mesmo tempo, instantâneos de memória salvos pela solução depois que os contêineres são removidos continuam no disco da máquina, podendo ser associados a sua origem (i.e., contêiner e imagem), conforme esperado para uma análise forense.
%
Essa capacidade se mantém após a detecção de uma ameaça, pois nesse caso coletas mais antigas deixam de ser apagadas.
%
Logo, é possível descrever o estado do sistema antes e depois do incidente \cite{CaseMemoryForensics:2014}, permitindo-se, por exemplo, que ataques de injeção de código em memória sejam analisados.



%\marcos{EVITE REDUNDÂNCIA ENTRE GRÁFICO E TABELA. Faz sentido apenas se um deles for trazer informações adicionais (e, nesses casos, em geral o gráfico/tabela acaba tendo algum highlight, para deixar a utilidade dessa redundância)
%\begin{table}[htb!]
%\centering
%\caption{Evolução do uso do espaço em disco}
%\label{tab:results-size}
%\begin{tabular}{c|c}
%\hline
%Tamanho total ocupado (KBytes) & Tempo (segundos) \\ \hline
%240                            & 1                \\ \hline
%480                            & 2                \\ \hline
%720                            & 3                \\ \hline
%960                            & 4                \\ \hline
%1200                           & 5                \\ \hline
%1200                           & 6                \\ \hline
%1200                           & 7                \\ \hline
%1200                           & 8                \\ \hline
%1200                           & 9                \\ \hline
%1200                           & 10                \\ \hline
%1200                           & 11                \\ \hline
%1200                           & 12                \\ \hline
%1200                           & 13                \\ \hline
%1200                           & 14                \\ \hline
%1200                           & 15                \\ \hline
%1200                           & 16                \\ \hline
%1200                           & 17                \\ \hline
%1200                           & 18                \\ \hline
%1200                           & 19                \\ \hline
%1200                           & 20                \\ \hline
%1200                           & 21                \\ \hline
%1200                           & 22                \\ \hline
%1200                           & 23                \\ \hline
%1200                           & 24                \\ \hline
%1200                           & 25                \\ \hline
%1200                           & 26                \\ \hline
%1200                           & 27                \\ \hline
%1200                           & 28                \\ \hline
%1200                           & 29                \\ \hline
%1200                           & 30                \\ \hline
%\end{tabular}
%\end{table}

\begin{figure}[htb!]
\footnotesize
\caption{Evolução do uso do espaço em disco com o Dizang}
\includegraphics[scale=0.60]{evolucao-coleta.pdf}
\centering
\label{fig:evolucao-coleta}
\begin{center}
Fonte: Próprio autor 
\end{center}
\end{figure}


\begin{comment}
A Figura \ref{fig:memoria_salva}, por sua vez, mostra uma listagem de alguns dos instantâneos de memória salvos pela solução depois que os contêineres são removidos. 
%
Nela pode-se ver que as coletas continuaram no disco da máquina mesmo após a remoção dos contêineres. 
%
Usando o identificador do contêiner e da imagem, consegue-se associar a evidência a sua origem (i.e., a imagem e o contêiner), conforme esperado para uma análise forense.
%
Essa capacidade se mantém após a detecção de uma ameaça, pois nesse caso coletas mais antigas deixam de ser apagadas.
%
Assim, é possível descrever o estado do sistema antes e depois do incidente \cite{Case_Memory_Forensics:2014}, permitindo-se, por exemplo, que ataques de injeção de código em memória sejam analisados.


\begin{figure*}[htb!]
\footnotesize
\caption{Exemplo de lista de instantâneos de memória.}
\fbox{
\includegraphics[scale=0.30]{memoria_salva.jpg}
}
\centering
\label{fig:memoria_salva}
\end{figure*}

\end{comment}

%No evento da detecção de uma ameaça a presente proposta deixa de apagar as coletas mais antigas. 
%
%Desta forma é capaz de descrever a história das alterações da memória do contêiner e com isso viabilizar a análise forense em busca das 4 vulnerabilidades de injeção de código em memória citadas no início do artigo. \marcos{Link bem fraco com introdução... pra que explicar em *detalhes* as vulnerabilidades na Introdução se você vai fazer uma explicação *superficial* de como elas são abordadas... Coloquei en passant para não dar a impressão de que você quis chamar a atenção para aquelas vulnerabilidades (sim, eu tinha pedido para você fazer esse link, mas um link tão fraco joga CONTRA você, não a favor...)}
%
%A viabilidade se dá pois consegue descrever o estado do sistema antes e depois do incidente \cite{Case_Memory_Forensics:2014}.
%

Uma potencial limitação da solução proposta é que a pausa de um contêiner para coleta de dados poder, em princípio, causar perdas no desempenho da aplicação sendo executada. 
%
Para avaliar esse impacto, durante o experimento foram medidos os tempos de cópia da memória do contêiner.
%
Os resultados são mostrados no gráfico da Figura \ref{fig:memoria-copia}.
%
É possível notar que, após a inicialização da aplicação, o tempo para realizar a cópia é bastante reduzido, variando entre 20 e 40 milissegundos. 
%
Em especial, para contêineres executando um motor de páginas web dinâmicas, como é o caso do experimento em questão, essa latência deve ser pouco perceptível por usuários finais.
%
Para os casos em que a interrupção da execução do recurso computacional mesmo por breves momentos cause problemas de disponibilidade, é possível realizar o procedimento de coleta em instantes de tempo separados.
%
Assim, ao invés de suspender a execução de todos os recursos computacionais para realização da coleta simultaneamente, o procedimento interrompe-as sequencialmente.
%
Desta forma, a latência demonstrada pode ser considerado o pior caso neste experimento.

\begin{figure}[htb!]
\footnotesize
\caption{Tempo de cópia da memória de um contêiner}
\includegraphics[scale=0.70]{memoria-copia.pdf}
\centering
\label{fig:memoria-copia}
\begin{center}
Fonte: Próprio autor 
\end{center}
\end{figure}


Outra preocupação é o tempo de transporte das evidências para o armazenamento fora da nuvem.
%
Caso o transporte da evidência leve mais tempo que a geração do próximo instantâneo, um backlog de transporte se formará levando a perdas nas evidências que estejam pendentes para transporte.
%
Para avaliar esse impacto, durante o experimento foram medidos os tempos de transporte das evidências para o armazenamento fora da nuvem.
%
Os resultados são mostrados no gráfico da Figura \ref{fig:evidencia_transporte}.
%
É possível notar que o tempo de transporte estabiliza após atingido o tamanho da janela. O tempo de transporte da evidência fica, em média próximo dos 30 segundos. 

%
Tanto a topologia quando a arquitetura do transporte da evidência e a arquitetura do que se deseja extrair a evidência são fatores que contribuem tanto positiva quando negativamente no tempo de transporte.
%
Neste experimento o gerador de evidências, um motor de páginas dinâmicas, está na América do Norte enquanto que a máquina física para onde as evidências foram transportadas e que é responsável pelo transporte da evidência está na América do Sul.

\begin{figure}[htb!]
\footnotesize
\caption{Tempo de transporte da evidência}
\includegraphics[scale=0.70]{evidencia-download.pdf}
\centering
\label{fig:evidencia_transporte}
\begin{center}
Fonte: Próprio autor 
\end{center}
\end{figure}


\subsection{Identificação de injeção de código malicioso}
\label{sec:proposta-exp-malware}

Um segundo experimento teve como objetivo determinar se é possível, através da análise das evidências coletadas, identificar injeção de código malicioso na memória do contêiner.
%
Para este fim uma biblioteca \textbf{libexample.so} simulando um código malicioso foi injetado em um dos contêineres.
%
Após cinco minutos de \fancyname realizando coletas, uma biblioteca foi injetada na memória de um dos contêineres. Após a injeção permitiu-se que a solução continuasse coletando por mais 5 minutos.
%
Além da coleta do conteúdo do diretório \textbf{/proc/pid/numa\_maps}, realizou-se também uma cópia crua da memória do processo do contêiner utilizando o utilitário \textit{nsenter} via comando descrito na Figura \ref{fig:comando-copia}.

\begin{figure}[htb!]
\footnotesize
\caption{Comando para cópia crua da memória do processo do contêiner}
\includegraphics[scale=0.60]{comando-copia-memoria-gdb.pdf}
\centering
\label{fig:comando-copia}
\begin{center}
Fonte: Próprio autor 
\end{center}
\end{figure}

%
De posse das coletas do diretório \textbf{/proc/pid/numa\_maps} comparou-se dois momentos distintos na vida do contêiner, antes e depois da injeção da biblioteca.
%
Observando as Figuras \ref{fig:antes-injecao} e \ref{fig:apos-injecao} é possível notar que no instantâneo após a injeção aparece a biblioteca \textbf{libexample.so} simulando o código malicioso entre os endereços \textbf{7f85631b8000} e \textbf{7f85633b9000}.
%
Logo, é possível identificar a injeção de um código malicioso via evidência coletada por \fancyname do diretório \textbf{/proc/pid/numa\_maps}, permitindo-se por exemplo que ataques de injeção de código sejam identificados.
%
A Figura \ref{fig:conteudo-memoria-copia-gdb} mostra o conteúdo da parte legível da memória no endereço \textbf{0x7f85633b9000} onde a biblioteca \textbf{libexample.so} simulando um código malicioso está alocada.

\begin{figure}[htb!]
\footnotesize
\caption{Parte do arquivo \textbf{/proc/pid/numa\_maps} ANTES da injeção }
\includegraphics[scale=0.80]{antes-injecao.pdf}
\centering
\label{fig:antes-injecao}
\begin{center}
Fonte: Próprio autor 
\end{center}
\end{figure}


\begin{figure}[htb!]
\footnotesize
\caption{Parte do arquivo \textbf{/proc/pid/numa\_maps} APÓS a injeção }
\includegraphics[scale=0.80]{apos-injecao.pdf}
\centering
\label{fig:apos-injecao}
\begin{center}
Fonte: Próprio autor 
\end{center}
\end{figure}

%
%As primeiras tentativas de cópia do conteúdo da memória do processo do contêiner foram feitas via \textit{ptrace} e não obteve sucesso. 
%
%Segundo \cite{cgroupsxptrace} isto ocorre pois as chamadas de sistema que ferramentas como \textit{ptrace} e \textit{htop} usam foram criadas antes da implementação de \textit{cgroups} no \textit{kernel} do linux e sendo assim não tem consciência da existência de isolamento entre processos.
%
%Quando o \textit{ptrace} tenta acessar uma área de memória isolada por \textit{cgroups}, o \textit{kernel} envia um sinal de violação de acesso de memória, o resultado é mostrado na Figura \ref{fig:erro-copia-gdb}.
%

%A documentação do Docker \cite{capabilities} menciona o comando \texttt{--cap-add=SYS_PTRACE --security-opt-seccomp=unconfined} que permite que o \textit{ptrace} consiga acessar a memória de um processo dentro do contêiner mas não permite que a máquina hospedeira ou outro contêiner tenha acesso (referência).
%
%Ainda segundo \cite{cgroupsxptrace} uma alternativa para viabilizar a monitoração e acesso a informações de memória seria o de expor tais informações na estrutura de \textbf{/sys/fs/cgroup/} da mesma forma que é feita para \textbf{/proc/pid/}.
%
%O sucesso na cópia do conteúdo da memória do processo do contêiner só foi alcançado quando utilizou-se a ferramenta \textit{nsenter}.
%

\begin{figure}[htb!]
\footnotesize
\caption{Conteúdo da memória de \textbf{libexample.so} no formato [endereço]: [conteúdo]}
\includegraphics[scale=0.65]{conteudo-memoria-copia-gdb.pdf}
\centering
\label{fig:conteudo-memoria-copia-gdb}
\begin{center}
Fonte: Próprio autor 
\end{center}
\end{figure}


%\begin{figure}[htb!]
%\footnotesize
%\caption{Tentativa mal sucedida de cópia do conteúdo da memória}
%\includegraphics[scale=0.65]{nao-consido-dumpar-memoria.png}
%\centering
%\label{fig:erro-copia-gdb}
%\begin{center}
%Fonte: Próprio autor 
%\end{center}
%\end{figure}


\section{Limitações}
\label{sec:proposta-limit}

A proposta descrita pede que o recurso em nuvem seja identificável de forma única a fim de realizar a associação entre evidência e sua origem.
%
Durante o curso deste projeto essa identificação única só foi possível através do hash da imagem do contêiner. Este foi o único recurso que, submetido ao processo de construção a partir da mesma receita resultou no mesmo hash da imagem.
%
Assim, a implementação para verificação da solução proposta consegue apenas coletar informações de memória no espaço do usuário (\textit{user space}), ela não consegue acessar o espaço de kernel (\textit{kernel space}). 
%
A implementação de \fancyname neste documento em princípio não consegue investigar códigos malicioso que se baseiam em informações do \textit{kernel space}.
%
Isso inclui, por exemplo, a comparação de informações do PEB ( \textit{Process Environment Block} -- Bloco para o Ambiente dos Processos ), que ficam no \textit{user space}, com informações do VAD ( \textit{Virtual Address Descriptor} -- Descritor de Endereços de Memória Virtual ), que fica no \textit{kernel space}. 
%
%Análise de ameaças do tipo DKOM ( \textit{Direct Kernel Object Manipulation} -- Manipulação Direta dos Objetos do Kernel ) também não se beneficiam com a solução aqui proposta. 
%\marcosT{Não entendi a ``associação com o contêiner'' aqui. Você quer dizer que ``não se beneficiam com a solução aqui proposta'' ou outra coisa? Você não definiu o que seria a tal ``associação com o contêiner'' fora do contexto da sua solução, então ficou confuso - Hamilton: Feito}.

%
Outra limitação da solução proposta é a necessidade da mesma estar instalada no sistema sob investigação a priori para que os resultados descritos neste documento sejam alcançados. 
%
Como mencionado em \cite{CaseMemoryForensics:2014}, ``para uma análise eficiente de um incidente em memória, são necessárias cópias da mesma \textbf{antes e depois} do incidente.''

\chapter{Conclusões e recomendações para trabalhos futuros}
\label{sec:proposta-concl-recom}

%
Neste capítulo são apresentadas as conclusões do presente trabalho e as recomendações para a continuidade dos trabalhos neste campo de estudo.

\subsection{Conclusões}
\label{sec:proposta-concl}

%\marcosT{Uma boa conclusão retoma, logo na primeira frase, o problema que ela se propunha a resolver. Comece com uma ou mais frases nesse sentido, (nota: \textbf{sem} copiar+colar de outro ponto do texto). - Hamilton: Feito} \marcosT{Só que não: eu disse para você retomar o \textbf{problema} na sua primeira frase. Sua frase começa retomando a \textbf{solução}. Essa que você colocou como primeira seria boa como uma segunda frase, em que você reforça como resolve o problema - Hamilton: Acho que agora foi :)}
%
Ameaças digitais que atuam diretamente na memória de sistema não costumam deixar rastros em disco após terem os recursos correspondentes desalocados, dificultando análises forenses posteriores.
%
Esse problema é especialmente notável em sistemas de computação em nuvem, nos quais a alocação e desalocação de recursos virtualizados (e.g., VMs e contêineres) é frequente.
%
Essa característica, aliada a aspectos como multi-inquilinato e multi-jurisdição de nuvens computacionais, dificulta a coleta de evidências para a investigação de incidentes.

%
Nesse cenário, a proposta apresentada visa relacionar o instantâneo de memória a sua origem, utilizando o \textit{hash} calculado do recurso computacional em nuvem como identificador da origem da evidência armazenada.
%
Para evitar uso excessivo de memória, a quantidade de dados armazenados usa uma janela de armazenamento, o que permite descrever a memória antes e depois de um ataque (e.g., de injeção de memória). 
%
Transportando de forma segura e armazenando a evidência em local conhecido fora da nuvem, evitam-se os problemas relacionados a multi-jurisdição e multi-inquilinato das nuvens computacionais.
%
A comparação de instantâneos de memória coletados em diferentes instantes de tempo permite a identificação de injeção de código assim como extrair o conteúdo da parte legível do endereço de memória correspondente.

%
Combinada com uma ferramenta para identificação de ameaças, essas características de \fancyname o transformam em uma solução poderosa para prover evidências e, assim, viabilizar análises forenses na nuvem.

\subsection{Trabalhos Futuros}
\label{sec: proposta-trab-fut}

%
Como mencionado em \ref{sec:proposta-limit} a solução proposta é capaz de gerar evidências de memória apenas do espaço do usuário (\textit{user space}). 
%
Alguns códigos malicioso injetados em memória são capazes de manipular o retorno de funções do kernel. Recomenda-se para trabalhos futuros a incorporação à \fancyname uma forma de realizar a extração do conteúdo de memória do espaço do kernel (\textit{kernel space}).


\bibliographystyle{abntex2-alf}
\bibliography{bibliography}

\appendix

%\input{appendices/conversion.tex}

\end{document}
