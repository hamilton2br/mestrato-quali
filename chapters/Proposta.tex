% ----------------------------------------------------------
% Proposta
% ----------------------------------------------------------
\chapter{Proposta de projeto: \fancyname}
\label{chp:proposta}

A presente proposta tem como objetivo principal coletar memória de recursos computacionais virtuais em arquitetura volátil de modo a conseguir: 
(1) identificar a fonte da evidência, mesmo se o recurso virtual não existir mais; 
(2) descrever o sistema antes e depois do incidente;
(3) transportar e armazenar a memória coletada de uma forma que garanta sua integridade e confidencialidade; e
(4) não violar a jurisdição e a privacidade de outros usuários que porventura tenham recursos alocados no mesmo servidor físico.
%
A solução aqui apresentada, denominada \fancyname, é descrita em detalhes a seguir.

\section{Identificação da origem}
\label{sec:proposal-desc-origin}

Em sistemas computacionais executados sobre uma infraestrutura física (i.e., não virtualizada), pode-se fazer uma associação direta entre um recurso qualquer e sua origem correspondente, seja este recurso uma informação da memória, imagem de disco ou pacotes trafegando na rede.
%
Já em sistemas construídos sobre uma infraestrutura virtual, em especial quando ela é auto-escalável, os recursos computacionais são altamente voláteis e, portanto, podem ser desalocados a qualquer momento.
%
Este fato torna difícil a associação de uma informação gerada por esta infraestrutura com sua origem.


Para conseguir correlacionar uma evidência a sua origem volátil, é necessário utilizar outro elemento em que persista a relação fonte-evidência.
%
O presente trabalho propõe que isto seja feito por meio de cálculo de hash do recurso em nuvem que produziu a evidência. %removendo contêiner para deixar mais genérico
%
%Embora um contêiner seja um \textit{software} e, portanto, também volátil, cada imagem compilada e sua execução na forma de contêiner são normalmente atrelados a um \textit{hash} que identifica univocamente essa relação. %removendo para deixar mais genérico
%
O hash de um recurso em nuvem permite identificar univocamente a fonte de uma evidência. Em arquiteturas que utilizam contêiner por exemplo, é possível identificar se a evidência veio do contêiner do motor de páginas dinâmicas (e.g., Apache%\cite{Tomcat}
), do contêiner da lógica de negócios (e.g., \textit{golang}%\cite{Google}
) ou do contêiner do banco de dados (e.g., \textit{Cassandra}%\cite{Cassandra}
). %removendo contêiner e deixando mais genérico

\section{Descrever o sistema antes e depois do incidente}
\label{sec:proposal-desc-incident}

%\marcosT{O parágrafo estava muito grande, então quebrei aqui. O problema é que falta uma frase para ligar a frase a seguir ao contexto da discussão... Coloque uma frase aqui, deixando claro qual requisito você quer satisfazer com essa ideia de ``interromper temporariamente a execução do contêiner''. Aplique isso para TUDO que for proposta: o leitor tem que saber de antemão pra que você está fazendo alguma coisa, ou vai ficar se perguntando ``Espera, mas pra que fazer isso?!'' - Hamilton: Feito}
A cópia de memória não é uma atividade atômica, pois ela é executada em conjunto com outros processos. 
%
Portanto, caso um desses processos seja um código malicioso apagando traços de sua existência da memória do recurso, informações possivelmente importantes para a investigação podem acabar sendo perdidas. 
%
Com o objetivo de deixar o processo de cópia da memória mais atômico, a fim de evitar inconsistências na informação coletada \cite{CaseMemoryForensics:2014}, \fancyname propõe que a execução do recurso em nuvem seja temporariamente suspenso para que seja realizada a cópia de sua memória. 
%
Essa técnica, que é semelhante àquela adotada em \cite{RafiqueStaticLiveDigitalForensics:2013} para VMs, produz um instantâneo da memória volátil do recurso; isso permite sua análise em um estado de repouso, ou seja, sem a necessidade de ter o recurso em execução.
%
Ao realizar a coleta em intervalos de tempo adequados, é possível construir um histórico do estado da memória durante a execução no recurso.
%
%\marcosT{A ligação entre as frases anterior e a seguir está bem ruim... você parece estar mudando completamente de assunto... Acredito que faltou uma frase dizendo que ``''salvar toda a memória'' pode se tornar um problema, reforçar isso com as duas frases que já estão a seguir, e depois dizer que você vai resolver.}


A maioria das técnicas forenses mais usadas atualmente são voltadas à obtenção da informação em sua totalidade.
%
Isso comumente é feito via cópia bit a bit ou por meio da obtenção do \textit{hardware} físico \cite{SimouCloudChlng:2014} \cite{BemPastPresentFuture:2008}. 
%
Embora tais técnicas possam parecer interessantes à primeira vista, elas muitas vezes acabam sendo responsáveis por um problema: o crescente volume de informações que os investigadores precisam analisar \cite{QuickIncreaseVolumeImpact:2014}.
%
Para mitigar essa dificuldade, em \fancyname são adotadas duas estratégias: a primeira é a definição de um volume de dados que possa ser considerado \textit{suficiente} para a realização de uma investigação; a segunda é a definição de uma \textit{idade máxima} para a evidência enquanto o sistema trabalha em condições normais, isto é, quando não está sob ataque.
%
Para detectar e analisar intrusões na memória de processos, é necessário ter uma cópia da memória antes e depois da intrusão \cite{CaseMemoryForensics:2014}. 
%
Assim, a solução proposta implementa uma janela de instantâneos de memória cobrindo um intervalo de tempo pré-definido, como ilustrado na Figura \ref{fig:janela}. 
%
Em condições normais de operação, as evidências são coletadas com certa periodicidade e coletas que atingem uma determinada idade são descartadas.
%
Em contraste, após a detecção de um evento de ataque (e.g., por um sistema de detecção de intrusões), \fancyname deixa de descartar as coletas mais antigas do \textit{log} de monitoramento.
%
Como resultado, é possível conhecer o sistema antes e depois do ataque e, assim, avaliar sua evolução.
%
%\marcosR{Não sei por que você está utilizando $\backslash\backslash$ no final das suas frases, mas pare de fazer isso... deixe o LaTeX se virar com a formatação. No final, quando você tiver tudo escrito, aí pode fazer algum sentido se preocupar com formatação, mas não antes disso... - Hamilton: é pra formatação mesmo :-) OK vou deixar o latex se virar}
%

\begin{figure}[htb!]
\footnotesize
\caption{Janela deslizante de coleta de evidência}
\includegraphics[scale=1.00]{janela.pdf}
\centering
\label{fig:janela}
\begin{center}
Fonte: Próprio autor 
\end{center}
\end{figure}

\section{Garantindo integridade, confidencialidade e protegendo privacidade e jurisdição}
\label{sec:proposal-desc-chain-of-custody}

%\marcosT{Essa frase não faz sentido: não se ``assina'' nada com um ``hash''. Você pode ``calcular o hash'' ou ``assinar um dado'' (e.g., um dado juntamente com o hash de alguma coisa). Revise essa frase... - Hamilton: Feito}
Finalmente, para persistir a relação evidência-origem e garantir a sua integridade, \fancyname calcula o hash $H$ do par \{evidência, identificador da imagem do contêiner\} e armazena a tripla \{$H$, identificador do recurso, evidência\}.
%
Adicionalmente, a presente proposta evita eventuais problemas com o armazenamento desses dados em países com jurisdições diferentes daquelas que devem ser aplicadas na investigação em questão.
%
Especificamente, as evidências coletadas são armazenadas em um local físico fora da nuvem, após serem transportadas por meio de um canal seguro (e.g., via TLS (\textit{Transport Layer Security} -- Camada de Transporte Seguro) \cite{DierksT2008}).
%

\section{Implementação}
\label{sec:proposta-impl}

%\marcosT{CLAREZA: quais ataques? Onde estão esses objetivos (diga a seção!!!)? Perceba que não tem qualquer seção com esse nome: você espera realmente que o leitor procure no seu texto onde eles estão...? - Hamilton: Feito}

\begin{figure}[htb!]
\footnotesize
\caption{Arquitetura geral da solução Dizang}
\includegraphics[scale=0.70]{Solucao.pdf}
\centering
\label{fig:Solucao}
\begin{center}
Fonte: Próprio autor 
\end{center}
\end{figure}

%
Os mecanismos propostos foram implementados em uma plataforma de testes visando avaliar a eficácia de \fancyname em coletar as informações de memória dos contêineres de forma reprodutível, sem violar jurisdições ou a privacidade de usuários e a capacidade de detectar injeção de código usando as evidências coletadas.
%
A solução, ilustrada na Figura \ref{fig:Solucao}, consistiu na criação de uma instancia \textit{t2.micro} na zona Ohio da AWS com 3.3Mhz, 1Gb de RAM e sistema operacional de 64 bits. 
%na criação de 1 VM usando o Oracle Virtual Box 5.0%\cite{VirtualBox} em um notebook Intel i5 de 2.30Mhz e 4Gb de RAM com sistema operacional de 64 bits.
%
Nesta instância AWS foi instalado o Docker Engine 1.10 e a API Docker 1.21, com os quais foram criados 3 contêineres executando o nginx 1.0 em diferentes portas. 
%
Foi desenvolvida uma aplicação Java que, executada no sistema operacional hospedeiro, descobre o identificador de processo associado a cada contêiner, copia o conteúdo do \textit{descritor de alocação de memória não uniforme} (\textbf{/proc/pid/numa\_maps}), o qual contém a alocação das páginas de memória, os nós que estão associados a essas páginas, o que está alocado e suas respectivas políticas de acesso \cite{UnixManPagesNumaMaps}.
%
A cópia e gravação do arquivo é tal que, a cada intervalo de tempo $t$, a aplicação (1) pausa o contêiner em questão, (2) copia a diretório \textbf{numa\_maps}, (3)  concatena os dados obtidos com o identificador da imagem e do contêiner, (4) calcula o $H$ do conjunto e (5) salva o resultado em um arquivo cujo nome é o identificador da imagem e do contêiner e a extensão é \textbf{.mem}. 
%
O transporte seguro da evidência para um armazenamento físico fora da AWS foi implementado usando uma instância \textit{t2.micro} na zona Ohio da AWS onde foi instalado um servidor \textit{OpenVPN}.
%
Como uma forma básica de controle de acesso, a instância EC2 que contém as evidências foi configurada para aceitar conexões apenas de máquinas nesta VPN.
%
Uma máquina física fora da AWS, usou o cliente do \textit{OpenVPN} para estabelecer uma conexão VPN com a instância que contém as evidências e as transportou para o disco da máquina física.
%
Após a conclusão do processo de transporte, a máquina física verifica se existem arquivos \textbf{.mem} em disco mais antigos que um certo intervalo de tempo $t$, descartando-os.
%


\section{Resultados experimentais}
\label{sec:proposta-exp}

Para avaliar a efetividade de \fancyname na coleta de evidências e identificação de injeção de código, dois experimentos foram realizados usando o ambiente implementado (descrito na Seção \ref{sec:proposta-impl}).
%


\subsection{Análise do desempenho}
\label{sec:proposta-exp-desempenho}

No primeiro experimento, o sistema foi configurado para realizar coletas de memória em intervalos de 1 minuto, salvá-las em armazenamento externo à nuvem e apagar amostras coletadas há mais de 5 minutos. 
%
O sistema foi então executado por 30 minutos, tempo durante o qual foram coletadas como métricas (1) o uso de espaço em disco utilizado pelos instantâneos de memória salvos, (2) o tempo de pausa no contêiner necessário para a cópia delas e (3) o tempo de transporte das evidências para o armazenamento externo a nuvem.


A evolução do espaço em disco ocupado pelos instantâneos de memória, acompanhado através da execução do comando \texttt{du -sh *.mem} do \textit{Unix} no disco de armazenamento externo, é mostrada no gráfico da Figura \ref{fig:evolucao-coleta}.
%
Neste experimento os instantâneos de memória tem 244kb de tamanho. 
%
O gráfico mostra que o aumento do uso do espaço em disco é linear e o crescimento se interrompe quando é atingido o limite de tempo configurado para a janela, pois as coletas com tempo de vida maior que tal limite são apagadas do disco. 
%
Assim, a solução mantém sob controle o espaço em disco ocupado pelas amostras coletadas.
%
Ao mesmo tempo, instantâneos de memória salvos pela solução depois que os contêineres são removidos continuam no disco da máquina, podendo ser associados a sua origem (i.e., contêiner e imagem), conforme esperado para uma análise forense.
%
Essa capacidade se mantém após a detecção de uma ameaça, pois nesse caso coletas mais antigas deixam de ser apagadas.
%
Logo, é possível descrever o estado do sistema antes e depois do incidente \cite{CaseMemoryForensics:2014}, permitindo-se, por exemplo, que ataques de injeção de código em memória sejam analisados.



%\marcos{EVITE REDUNDÂNCIA ENTRE GRÁFICO E TABELA. Faz sentido apenas se um deles for trazer informações adicionais (e, nesses casos, em geral o gráfico/tabela acaba tendo algum highlight, para deixar a utilidade dessa redundância)
%\begin{table}[htb!]
%\centering
%\caption{Evolução do uso do espaço em disco}
%\label{tab:results-size}
%\begin{tabular}{c|c}
%\hline
%Tamanho total ocupado (KBytes) & Tempo (segundos) \\ \hline
%240                            & 1                \\ \hline
%480                            & 2                \\ \hline
%720                            & 3                \\ \hline
%960                            & 4                \\ \hline
%1200                           & 5                \\ \hline
%1200                           & 6                \\ \hline
%1200                           & 7                \\ \hline
%1200                           & 8                \\ \hline
%1200                           & 9                \\ \hline
%1200                           & 10                \\ \hline
%1200                           & 11                \\ \hline
%1200                           & 12                \\ \hline
%1200                           & 13                \\ \hline
%1200                           & 14                \\ \hline
%1200                           & 15                \\ \hline
%1200                           & 16                \\ \hline
%1200                           & 17                \\ \hline
%1200                           & 18                \\ \hline
%1200                           & 19                \\ \hline
%1200                           & 20                \\ \hline
%1200                           & 21                \\ \hline
%1200                           & 22                \\ \hline
%1200                           & 23                \\ \hline
%1200                           & 24                \\ \hline
%1200                           & 25                \\ \hline
%1200                           & 26                \\ \hline
%1200                           & 27                \\ \hline
%1200                           & 28                \\ \hline
%1200                           & 29                \\ \hline
%1200                           & 30                \\ \hline
%\end{tabular}
%\end{table}

\begin{figure}[htb!]
\footnotesize
\caption{Evolução do uso do espaço em disco com o Dizang}
\includegraphics[scale=0.60]{evolucao-coleta.pdf}
\centering
\label{fig:evolucao-coleta}
\begin{center}
Fonte: Próprio autor 
\end{center}
\end{figure}


\begin{comment}
A Figura \ref{fig:memoria_salva}, por sua vez, mostra uma listagem de alguns dos instantâneos de memória salvos pela solução depois que os contêineres são removidos. 
%
Nela pode-se ver que as coletas continuaram no disco da máquina mesmo após a remoção dos contêineres. 
%
Usando o identificador do contêiner e da imagem, consegue-se associar a evidência a sua origem (i.e., a imagem e o contêiner), conforme esperado para uma análise forense.
%
Essa capacidade se mantém após a detecção de uma ameaça, pois nesse caso coletas mais antigas deixam de ser apagadas.
%
Assim, é possível descrever o estado do sistema antes e depois do incidente \cite{Case_Memory_Forensics:2014}, permitindo-se, por exemplo, que ataques de injeção de código em memória sejam analisados.


\begin{figure*}[htb!]
\footnotesize
\caption{Exemplo de lista de instantâneos de memória.}
\fbox{
\includegraphics[scale=0.30]{memoria_salva.jpg}
}
\centering
\label{fig:memoria_salva}
\end{figure*}

\end{comment}

%No evento da detecção de uma ameaça a presente proposta deixa de apagar as coletas mais antigas. 
%
%Desta forma é capaz de descrever a história das alterações da memória do contêiner e com isso viabilizar a análise forense em busca das 4 vulnerabilidades de injeção de código em memória citadas no início do artigo. \marcos{Link bem fraco com introdução... pra que explicar em *detalhes* as vulnerabilidades na Introdução se você vai fazer uma explicação *superficial* de como elas são abordadas... Coloquei en passant para não dar a impressão de que você quis chamar a atenção para aquelas vulnerabilidades (sim, eu tinha pedido para você fazer esse link, mas um link tão fraco joga CONTRA você, não a favor...)}
%
%A viabilidade se dá pois consegue descrever o estado do sistema antes e depois do incidente \cite{Case_Memory_Forensics:2014}.
%

Uma potencial limitação da solução proposta é que a pausa de um contêiner para coleta de dados poder, em princípio, causar perdas no desempenho da aplicação sendo executada. 
%
Para avaliar esse impacto, durante o experimento foram medidos os tempos de cópia da memória do contêiner.
%
Os resultados são mostrados no gráfico da Figura \ref{fig:memoria-copia}.
%
É possível notar que, após a inicialização da aplicação, o tempo para realizar a cópia é bastante reduzido, variando entre 20 e 40 milissegundos. 
%
Em especial, para contêineres executando um motor de páginas web dinâmicas, como é o caso do experimento em questão, essa latência deve ser pouco perceptível por usuários finais.
%
Para os casos em que a interrupção da execução do recurso computacional mesmo por breves momentos cause problemas de disponibilidade, é possível realizar o procedimento de coleta em instantes de tempo separados.
%
Assim, ao invés de suspender a execução de todos os recursos computacionais para realização da coleta simultaneamente, o procedimento interrompe-as sequencialmente.
%
Desta forma, a latência demonstrada pode ser considerado o pior caso neste experimento.

\begin{figure}[htb!]
\footnotesize
\caption{Tempo de cópia da memória de um contêiner}
\includegraphics[scale=0.70]{memoria-copia.pdf}
\centering
\label{fig:memoria-copia}
\begin{center}
Fonte: Próprio autor 
\end{center}
\end{figure}


Outra preocupação é o tempo de transporte das evidências para o armazenamento fora da nuvem.
%
Caso o transporte da evidência leve mais tempo que a geração do próximo instantâneo, um backlog de transporte se formará levando a perdas nas evidências que estejam pendentes para transporte.
%
Para avaliar esse impacto, durante o experimento foram medidos os tempos de transporte das evidências para o armazenamento fora da nuvem.
%
Os resultados são mostrados no gráfico da Figura \ref{fig:evidencia_transporte}.
%
É possível notar que o tempo de transporte estabiliza após atingido o tamanho da janela. O tempo de transporte da evidência fica, em média próximo dos 30 segundos. 

%
Tanto a topologia quando a arquitetura do transporte da evidência e a arquitetura do que se deseja extrair a evidência são fatores que contribuem tanto positiva quando negativamente no tempo de transporte.
%
Neste experimento o gerador de evidências, um motor de páginas dinâmicas, está na América do Norte enquanto que a máquina física para onde as evidências foram transportadas e que é responsável pelo transporte da evidência está na América do Sul.

\begin{figure}[htb!]
\footnotesize
\caption{Tempo de transporte da evidência}
\includegraphics[scale=0.70]{evidencia-download.pdf}
\centering
\label{fig:evidencia_transporte}
\begin{center}
Fonte: Próprio autor 
\end{center}
\end{figure}


\subsection{Identificação de injeção de código malicioso}
\label{sec:proposta-exp-malware}

Um segundo experimento teve como objetivo determinar se é possível, através da análise das evidências coletadas, identificar injeção de código malicioso na memória do contêiner.
%
Para este fim uma biblioteca \textbf{libexample.so} simulando um código malicioso foi injetado em um dos contêineres.
%
Após cinco minutos de \fancyname realizando coletas, uma biblioteca foi injetada na memória de um dos contêineres. Após a injeção permitiu-se que a solução continuasse coletando por mais 5 minutos.
%
Além da coleta do conteúdo do diretório \textbf{/proc/pid/numa\_maps}, realizou-se também uma cópia crua da memória do processo do contêiner utilizando o utilitário \textit{nsenter} via comando descrito na Figura \ref{fig:comando-copia}.

\begin{figure}[htb!]
\footnotesize
\caption{Comando para cópia crua da memória do processo do contêiner}
\includegraphics[scale=0.60]{comando-copia-memoria-gdb.pdf}
\centering
\label{fig:comando-copia}
\begin{center}
Fonte: Próprio autor 
\end{center}
\end{figure}

%
De posse das coletas do diretório \textbf{/proc/pid/numa\_maps} comparou-se dois momentos distintos na vida do contêiner, antes e depois da injeção da biblioteca.
%
Observando as Figuras \ref{fig:antes-injecao} e \ref{fig:apos-injecao} é possível notar que no instantâneo após a injeção aparece a biblioteca \textbf{libexample.so} simulando o código malicioso entre os endereços \textbf{7f85631b8000} e \textbf{7f85633b9000}.
%
Logo, é possível identificar a injeção de um código malicioso via evidência coletada por \fancyname do diretório \textbf{/proc/pid/numa\_maps}, permitindo-se por exemplo que ataques de injeção de código sejam identificados.
%
A Figura \ref{fig:conteudo-memoria-copia-gdb} mostra o conteúdo da parte legível da memória no endereço \textbf{0x7f85633b9000} onde a biblioteca \textbf{libexample.so} simulando um código malicioso está alocada.

\begin{figure}[htb!]
\footnotesize
\caption{Parte do arquivo \textbf{/proc/pid/numa\_maps} ANTES da injeção }
\includegraphics[scale=0.80]{antes-injecao.pdf}
\centering
\label{fig:antes-injecao}
\begin{center}
Fonte: Próprio autor 
\end{center}
\end{figure}


\begin{figure}[htb!]
\footnotesize
\caption{Parte do arquivo \textbf{/proc/pid/numa\_maps} APÓS a injeção }
\includegraphics[scale=0.80]{apos-injecao.pdf}
\centering
\label{fig:apos-injecao}
\begin{center}
Fonte: Próprio autor 
\end{center}
\end{figure}

%
%As primeiras tentativas de cópia do conteúdo da memória do processo do contêiner foram feitas via \textit{ptrace} e não obteve sucesso. 
%
%Segundo \cite{cgroupsxptrace} isto ocorre pois as chamadas de sistema que ferramentas como \textit{ptrace} e \textit{htop} usam foram criadas antes da implementação de \textit{cgroups} no \textit{kernel} do linux e sendo assim não tem consciência da existência de isolamento entre processos.
%
%Quando o \textit{ptrace} tenta acessar uma área de memória isolada por \textit{cgroups}, o \textit{kernel} envia um sinal de violação de acesso de memória, o resultado é mostrado na Figura \ref{fig:erro-copia-gdb}.
%

%A documentação do Docker \cite{capabilities} menciona o comando \texttt{--cap-add=SYS_PTRACE --security-opt-seccomp=unconfined} que permite que o \textit{ptrace} consiga acessar a memória de um processo dentro do contêiner mas não permite que a máquina hospedeira ou outro contêiner tenha acesso (referência).
%
%Ainda segundo \cite{cgroupsxptrace} uma alternativa para viabilizar a monitoração e acesso a informações de memória seria o de expor tais informações na estrutura de \textbf{/sys/fs/cgroup/} da mesma forma que é feita para \textbf{/proc/pid/}.
%
%O sucesso na cópia do conteúdo da memória do processo do contêiner só foi alcançado quando utilizou-se a ferramenta \textit{nsenter}.
%

\begin{figure}[htb!]
\footnotesize
\caption{Conteúdo da memória de \textbf{libexample.so} no formato [endereço]: [conteúdo]}
\includegraphics[scale=0.65]{conteudo-memoria-copia-gdb.pdf}
\centering
\label{fig:conteudo-memoria-copia-gdb}
\begin{center}
Fonte: Próprio autor 
\end{center}
\end{figure}


%\begin{figure}[htb!]
%\footnotesize
%\caption{Tentativa mal sucedida de cópia do conteúdo da memória}
%\includegraphics[scale=0.65]{nao-consido-dumpar-memoria.png}
%\centering
%\label{fig:erro-copia-gdb}
%\begin{center}
%Fonte: Próprio autor 
%\end{center}
%\end{figure}


\section{Limitações}
\label{sec:proposta-limit}

A proposta descrita pede que o recurso em nuvem seja identificável de forma única a fim de realizar a associação entre evidência e sua origem.
%
Durante o curso deste projeto essa identificação única só foi possível através do hash da imagem do contêiner. Este foi o único recurso que, submetido ao processo de construção a partir da mesma receita resultou no mesmo hash da imagem.
%
Assim, a implementação para verificação da solução proposta consegue apenas coletar informações de memória no espaço do usuário (\textit{user space}), ela não consegue acessar o espaço de kernel (\textit{kernel space}). 
%
A implementação de \fancyname neste documento em princípio não consegue investigar códigos malicioso que se baseiam em informações do \textit{kernel space}.
%
Isso inclui, por exemplo, a comparação de informações do PEB ( \textit{Process Environment Block} -- Bloco para o Ambiente dos Processos ), que ficam no \textit{user space}, com informações do VAD ( \textit{Virtual Address Descriptor} -- Descritor de Endereços de Memória Virtual ), que fica no \textit{kernel space}. 
%
%Análise de ameaças do tipo DKOM ( \textit{Direct Kernel Object Manipulation} -- Manipulação Direta dos Objetos do Kernel ) também não se beneficiam com a solução aqui proposta. 
%\marcosT{Não entendi a ``associação com o contêiner'' aqui. Você quer dizer que ``não se beneficiam com a solução aqui proposta'' ou outra coisa? Você não definiu o que seria a tal ``associação com o contêiner'' fora do contexto da sua solução, então ficou confuso - Hamilton: Feito}.

%
Outra limitação da solução proposta é a necessidade da mesma estar instalada no sistema sob investigação a priori para que os resultados descritos neste documento sejam alcançados. 
%
Como mencionado em \cite{CaseMemoryForensics:2014}, ``para uma análise eficiente de um incidente em memória, são necessárias cópias da mesma \textbf{antes e depois} do incidente.''

\chapter{Conclusões e recomendações para trabalhos futuros}
\label{sec:proposta-concl-recom}

%
Neste capítulo são apresentadas as conclusões do presente trabalho e as recomendações para a continuidade dos trabalhos neste campo de estudo.

\subsection{Conclusões}
\label{sec:proposta-concl}

%\marcosT{Uma boa conclusão retoma, logo na primeira frase, o problema que ela se propunha a resolver. Comece com uma ou mais frases nesse sentido, (nota: \textbf{sem} copiar+colar de outro ponto do texto). - Hamilton: Feito} \marcosT{Só que não: eu disse para você retomar o \textbf{problema} na sua primeira frase. Sua frase começa retomando a \textbf{solução}. Essa que você colocou como primeira seria boa como uma segunda frase, em que você reforça como resolve o problema - Hamilton: Acho que agora foi :)}
%
Ameaças digitais que atuam diretamente na memória de sistema não costumam deixar rastros em disco após terem os recursos correspondentes desalocados, dificultando análises forenses posteriores.
%
Esse problema é especialmente notável em sistemas de computação em nuvem, nos quais a alocação e desalocação de recursos virtualizados (e.g., VMs e contêineres) é frequente.
%
Essa característica, aliada a aspectos como multi-inquilinato e multi-jurisdição de nuvens computacionais, dificulta a coleta de evidências para a investigação de incidentes.

%
Nesse cenário, a proposta apresentada visa relacionar o instantâneo de memória a sua origem, utilizando o \textit{hash} calculado do recurso computacional em nuvem como identificador da origem da evidência armazenada.
%
Para evitar uso excessivo de memória, a quantidade de dados armazenados usa uma janela de armazenamento, o que permite descrever a memória antes e depois de um ataque (e.g., de injeção de memória). 
%
Transportando de forma segura e armazenando a evidência em local conhecido fora da nuvem, evitam-se os problemas relacionados a multi-jurisdição e multi-inquilinato das nuvens computacionais.
%
A comparação de instantâneos de memória coletados em diferentes instantes de tempo permite a identificação de injeção de código assim como extrair o conteúdo da parte legível do endereço de memória correspondente.

%
Combinada com uma ferramenta para identificação de ameaças, essas características de \fancyname o transformam em uma solução poderosa para prover evidências e, assim, viabilizar análises forenses na nuvem.

\subsection{Trabalhos Futuros}
\label{sec: proposta-trab-fut}

%
Como mencionado em \ref{sec:proposta-limit} a solução proposta é capaz de gerar evidências de memória apenas do espaço do usuário (\textit{user space}). 
%
Alguns códigos malicioso injetados em memória são capazes de manipular o retorno de funções do kernel. Recomenda-se para trabalhos futuros a incorporação à \fancyname uma forma de realizar a extração do conteúdo de memória do espaço do kernel (\textit{kernel space}).