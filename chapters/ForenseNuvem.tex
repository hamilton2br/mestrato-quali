% ----------------------------------------------------------
% Capitulo de Forense em nuvem
% ----------------------------------------------------------
\chapter{Forense de memória de máquinas em nuvem}

\section{Credibilidade e aceitabilidade da evidência em processo legal.}
\label{sec:credibilidadeaceitabilidadeevidencia}

O processo de análise forense no evento de um crime digital é descrito no EDIPM - \textit{Enhanced Digital Investigation Process Model} por 4 fases: Identificar, preservar, examinar, apresentar. \cite{GrisposChallengesCloudComputing:2012}
%
Na fase de preservação da evidência deve ser feita de forma que os autores descrevem como ``forensicamente aceitável'' isto é, coletar as evidências de forma que as mesmas sejam aceitas em um processo legal e não sejam invalidadas durante o processo.
%
Segundo \cite{Ramos:2011} a aceitabilidade de uma evidência digital em um processo legal deve atender aos seguintes requisitos: Autenticidade Processo pelo qual se pode garantir a autoria do documento eletrônico, ou seja, não permite dúvida quanto à identificação do autor.
%
e Integridade: Permite atestar a “inteireza do documento eletrônico após sua transmissão, bem como apontar eventual alteração irregular de seu conteúdo”.
%
Caso haja dúvida sob qualquer um dos requisitos uma perícia técnica pode ser convocada, nesta será análisada o autor da evidência ou seja sua fonte e se a mesma não foi alterada no processo.
%
Em infra-estrutura física esta coleta era relativamente simples, bastava-se remover o recurso físico, transporta-lo para um laboratório e lá análisar a evidência. A Evidência era mantida em uma sala-cofre onde o acesso era controlado.
%
A reprodutibilidade do processo de coleta e a manutenção da integridade da evidência eram tarefas bem diretas.
%
A computação em núvem, especialmente as de infraestrutura auto-escalável troxeram um conjunto de desafios para se atingir este requisito. O recurso não pode mais ser removido pois o mesmo é utilizado por outros usuários não relacionados a investigação, fazê-lo constituiria violação de privacidade.
%
A volatilidade dos recursos tornou a verificação do seu autor um processo mais complexo pois o recurso que a gerou pode não existir mais.
%
A integridade da evidência também tornou-se mais complexa pois ela precisa ser coletada, transportada e armazenada. O processo de cadeia de custôdia ganhou grande visibilidade neste quesito.
%
Violação de qualquer uma das caracteristicas citadas anteriormente põe em dúvida a credibilidade da evidência.



