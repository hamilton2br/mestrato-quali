% ----------------------------------------------------------
% Capitulo de Forense em nuvem
% ----------------------------------------------------------
\chapter{Forense de memória de máquinas em nuvem}

\section{Credibilidade e aceitabilidade da evidência em processo legal.}
\label{sec:credibilidadeaceitabilidadeevidencia}

O processo de análise forense no evento de um crime digital é descrito no EDIPM - \textit{Enhanced Digital Investigation Process Model} por 4 fases: Identificar, preservar, examinar, apresentar. (calm before the storm)
%
Na fase de preservação da evidência deve ser feita de forma que os autoras descrevem como ``forensicamente aceitável'' isto é, coletar as evidências de forma que as mesmas sejam aceitas em um processo legal e não sejam invalidadas durante o processo.
%
Segundo (valor probatório do aquivo digital) a aceitabilidade de uma evidência digital em um processo legal deve atender aos seguintes requisitos: Autenticidade Processo pelo qual se pode garantir a autoria do documento eletrônico, ou seja, não permite dúvida quanto à identificação do autor.
%
e Integridade: Permite atestar a “inteireza do documento eletrônico após sua transmissão, bem como apontar eventual alteração irregular de seu conteúdo”.
%
Caso haja dúvida sob qualquer um dos requisitos uma perícia técnica pode ser convocada, nesta sera análisada o autor da evidência ou seja sua fonte e se a mesma não foi alterada no processo.
%
Em infra-estrutura física esta coleta era relativamente simples, bastava-se remover o recurso físico, transporta-lo para um laboratório e lá análisar a evidência. A Evidência era mantida em uma sala-cofre onde o acesso era controlado.
%
A reprodutibilidade do processo de coleta e a manutenção da integridade da evidência eram tarefas bem diretas.
%
A computação em núvem, especialmente as de infraestrutura auto-escalável troxeram um conjunto de desafios para se atingir este requisito. O recurso não pode mais ser removido pois o mesmo é utilizado por outros usuários não relacionados a investigação, fazê-lo constituiria violação de privacidade.
%
A volatilidade dos recursos tornou a verificação do seu autor um processo mais complexo pois o recurso que a gerou pode não existir mais.
%
A integridade da evidência também tornou-se mais complexa pois ela precisa ser coletada, transportada e armazenada. O processo de cadeia de custôdia ganhou grande visibilidade neste quesito.
%
Violação de qualquer uma das caracteristicas citadas anteriormente põe em dúvida a credibilidade da evidência.

\section{Volume de dados para coleta}
\label{sec:volumedados}

O processo de coleta da evidência na forense digital herdou suas práticas da forense tradicional onde isola-se cena do crime e coletam-se as evidências presentes. 
%
Transportando para a forense digital criou-se o hábito de se realizar cópia bit a bit da informação que se deseja investigar. 
%
No passado, com as soluções tendo bem menos capacidade de memória, disco e tráfego, tal prática não trazia problemas. Nas atuais soluções, aplicações e arquiteturas em núvem o volume de dados aumentou  muito.
%
Em (encontrar a data) investigadores forenses tinham em média 6 meses de backlog para analisar. Em conversas informais com analistas forenses é comum a métrica de em média apenas 2\% do material coletado ser útil a análise.
%
Encontrar uma forma de armazenar menos informações de modo a tornar a fase de análise mais rápida e eficiênte ajudará nas investigações.

\section{Privacidade e jurisdição}
\label{sec:violacaoprivacidadejuriscdicao}

Na metodologia tradicional de coleta de evidências para análise isola-se o ambiente e as evidencias são removidas. Transportado para a forense digital temos a prática de remover o equipamento para realização de cópia bit a bit da evidência.
%
Nas soluções de infra estrutura física esta prática não trás problemas, os objeto ou indivíduos sob investigação estão diretamente relacionados ao equipamento removido. 
%
Nas soluções em nuvem esta prática não pode mais ser utilizada pois como o recurso físico é compartilhado por vários usuários não envolvidos na investigação, remove-los configura violação de privacidade.
%
Um complicador a mais é o fato de os dados não estarem armazenados no mesmo território em que a investigação é realizada demandando acordos de cooperação jurídica entre as partes o que nem sempre é possível.
%
Neste cenário encontrar uma forma de coletar a evidência sem violar jurisdição e privacidade são de grande importância num futuro próximo.

