% ----------------------------------------------------------
% Capitulo de Aspectos conceituais e economigos 
% ----------------------------------------------------------
\chapter{Fundamentação Teórica}
\label{chp:fundamentação}


Desde 2001, diversos modelos para a condução de investigação digital foram propostos. 
%
O \textit{Enhanced Digital Investigation Process Model}, proposto por Carrier e Stafford em 2003, é a última iteração na evolução do processo forense digital \cite{GrisposChallengesCloudComputing:2012}. \marcos{TODO: Precisa da referência completa na seção de referências e uma citação aqui... - Hamilton: Feito}
%
Entretanto, como tais modelos de investigação foram desenvolvidos antes da aparição de tecnologias de computação em nuvem, muitos partem do pressuposto que o investigador tem acesso e controle sobre o sistema sob investigação \cite{GrisposChallengesCloudComputing:2012}.
%
Esta defasagem é um dos desafios da forense digital atual.


Neste capítulo, são discutidos os principais conceitos que permeiam esse cenário.
%
Especificamente, após uma discussão geral sobre computação em nuvem e suas particularidades, incluindo o uso de contêineres, são discutidas os características esperadas de um processo de forense digital robusto.


\section{Nuvens computacionais e contêineres}
\label{sec:computacaonuvem}

Uma nuvem computacional é um modelo de infraestrutura no qual recursos compartilhados em quantidade configurável, acessíveis via rede, são alocados e desalocados com esforço mínimo de gerenciamento por parte de um provedor de serviços.
%
Existem basicamente três modelos principais de comercialização de uso da nuvem \cite{NIST2011}: 

\begin{itemize}
	\item SaaS ( \textit{Software as a Service} -- \textit{Software} como serviço ): Onde se provê o \textit{software} que será utilizado; nesse caso, os clientes do serviço são os usuários finais o software.
	
	\item PaaS ( \textit{Platform as a Service} -- Plataforma como serviço ): Onde se provê o ambiente para o desenvolvimento, teste e execução do \textit{software}; nesse caso, os clientes do serviço são desenvolvedores de aplicações.
	
	\item IaaS ( \textit{Infrastructure as a Service} -- Infraestrutura como serviço ): Onde são fornecidos recursos computacionais básicos, como processamento, memória e redes, em geral de forma virtualizada; os clientes desse tipo de serviço costumam ser arquitetos de sistemas.
\end{itemize}

O tipo de serviço de nuvem mais pertinente para este trabalho é o IaaS, uma solução muito usada atualmente pela sua capacidade de prover recursos sob demanda de forma auto-escalável.
%
Nesse cenário, o uso intenso de tecnologias de virtualização costuma levar a recursos altamente voláteis, que são alocados e desalocados a qualquer momento pelo orquestrador da nuvem para suprir um suprir eventuais aumentos e reduções de demanda.
%
É possível até mesmo construir scripts para a automatizar a construção da arquitetura desejada, permitindo a instanciação e interconexão de máquinas adequadas para a atividade fim do sistema \marcos{citar_pupper+chef_ou_outras tecnologias_para_esse_fim - Hamilton: Feito}.
%
Esses scripts, escritos em linguagem específica como Puppet \cite{Puppet2018}, Chef \cite{Chef2018} e Vagrant \cite{Vagrant2018}, podem conter instruções de como distribuir o tráfego de rede entre as diferentes instâncias de computação ou armazenamento.
%
Dentre as vantagens de sua utilização, podem ser citadas a capacidade de usar os recursos de nuvem de uma forma mais eficiente, levar a uma menor necessidade de intervenção humana, e prover maior resiliência a variações de demanda do sistema.


%\section{Uso de contêineres}
%\label{sec:conteiner}

Uma tecnologia de virtualização possível para cenários de nuvem, e cuja utilização vem crescendo nos últimos anos, são os chamados contêineres \cite{containers-tech:2014}. 
%
Basicamente, um contêiner é um método de virtualização do sistema operacional que permite executar uma aplicação, bem como suas dependências, em um processo no qual recursos como disco, memória e rede permanecem isolados.
%
Diferente das máquinas virtuais, a virtualização com contêineres é feita no nível do Sistema Operacional (SO) nativo.
%
Como resultado, tem-se uma implementação de virtualização na qual eliminam-se camadas entre o aplicativo executado e o \textit{hardware} físico, permitindo maior granularidade no controle sobre esses recursos e melhorando a eficiência da infraestrutura.


Uma implementação bastante utilizada para esse propósito são os LXC ( \textit{LinuX Conteiners} -- Contêineres Linux ) \cite{Linuxcontainers.org2015}, que aproveitam-se de funcionalidades como \textit{cgroups}, \textit{kernel namespacing} e \textit{chroot} do kernel do Linux para auxiliar no gerenciamento e isolamento de recursos virtuais.
%
Mais precisamente, a funcionalidade de \textit{cgroups} ( \textit{Control Groups} -- Grupos de Controle ) presente no Kernel do Linux limita e isola o uso de recursos como CPU, memória e disco de um conjunto de processos, além de organizá-los de forma hierárquica. 
%
O trabalho nessa funcionalidade começou em 2006 na Google, sob a denominação de \textit{process container}. 
%
No final de 2007, seu nome foi alterado para \textit{control groups}, e o resultado foi então adicionado à versão 2.6.24 do kernel lançado em 2008 \cite{UnixManPagesControlGroups}.

Já o \textit{Namespacing} é uma funcionalidade do Kernel do Linux usada para isolar e virtualizar recursos do sistema operacional, como identificadores de processos, acessos à rede, comunicação inter-processos e sistema de arquivos.
%
\textit{Namespacing} envolve os recursos do sistema operacional em uma abstração que faz parecer aos processos de um mesmo \textit{namespace} que eles tem sua própria instância isolada de um recurso global.
%
Desta forma, essa é a principal funcionalidade por trás da implementação de Contêineres Linux \cite{UnixManPagesNamespacing}.


Finalmente, \textit{chroot} ( \textit{Change Root} -- Mude Root ) é uma funcionalidade do Kernel do Linux usada para mudar o diretório \textit{root} enxergado pelo processo que está chamando a função, bem como por todos os seus processos filhos. 
%
A chamada a \textit{chroot} altera o processo de resolução de caminhos do sistema operacional para o processo que o chamou \cite{UnixManPagesChRoot}.
%
Desta forma, pode-se instalar uma distribuição Linux secundária em uma pasta, ao invés de uma partição, e executar programas desta pasta sem perda significativa de desempenho.


\section{Forense digital e seus desafios}
\label{sec:forensedigital}


A área de forense digital (também conhecida por forense computacional) refe-se a um conjunto de técnicas de coleta e análise da interação entre humanos e computadores de forma que esta seja aceita em um processo legal.
%
Tal como a forense tradicional, a forense digital se baseia no princípio de Locard, estabelecido pelo médico francês Edmond Locard da seguinte forma: ``Quando um indivíduo entra em contato com outro objeto ou indivíduo, este sempre deixa vestígio deste contato'' \cite{Ramos:2011}.
%
De forma similar, a forense digital tem por objetivo a investigação de evidências digitais da interação entre homem e máquina, de modo a reconstruir a cadeia de eventos passados para que suas conclusões sejam validadas por terceiros e sejam aceitas em um processo legal.
%

A seguir são detalhados os principais desafios enfrentados pela forense digital quando aplicada a infraestruturas em nuvem pública.
\marcos{senti falta de uma frase aqui para preparar o leitor para as sub-seções a seguir. O que são as sub-seções? Princípios que devem ser seguidos pela forense digital? Desafios da forense digital? Outra coisa? Inclua uma frase aqui para fazer esse ``gancho'' - Hamilton: feito}

\subsection{Aceitabilidade da evidência em processo legal.}
\label{sec:credibilidadeaceitabilidadeevidencia}

O processo de análise forense no evento de um crime digital é descrito no EDIPM ( \textit{Enhanced Digital Investigation Process Model} -- Modelo Melhorado para Processo de Investigação Digital )\marcos{Sigla sem colocar por extenso o significado = Erro comum 40. Revise o texto INTEIRO por esse erro pfv. - Hamilton: Feito} por 4 fases \cite{GrisposChallengesCloudComputing:2012}: identificar, preservar, examinar e apresentar.
%
A fase mais pertinente a este trabalho é a de preservação da evidência, que deve ser conduzida de forma forensicamente aceitável.
%
Ou seja, deve-se coletar as evidências de forma que elas sejam aceitas em um processo legal e não sejam invalidadas no curso do mesmo.


Para atingir tal objetivo, o primeiro passo é a garantia da cadeia de custódia relacionada a evidência.
%
Cadeia de custódia é o processo de documentação da história cronológica da evidência de modo a saber onde a evidência esteve e quem teve acesso a ela \cite{Ramos:2011}. 
%
%A SENASP (Secretaria Nacional de Segurança Pública) \marcos{Que sigla é essa? Cadê a referência? - Hamilton: removido, repetitivo} define cadeia de custódia como ``a sistemática de procedimentos que visa à preservação do valor probatório da prova pericial caracterizada.''


O passo seguinte é a garantia da autenticidade e da integridade da evidência.
%
Autenticidade pode ser definida como ``o processo pelo qual se pode garantir a autoria do documento eletrônico'', ou seja, por meio do qual se não permite dúvida quanto à identificação do autor \cite{Ramos:2011}. \marcos{Se usou aspas, é porque copiou de algum lugar. Nesse caso, faltou a referência (é ``\cite{Ramos:2011}''?). Se não copiou de algum lugar, então o uso de aspas está incorreto. - Hamilton : Feito}.
%
Já a integridade pode ser definida como ``o atestado da inteireza do documento eletrônico após sua transmissão, bem como apontar eventual alteração irregular de seu conteúdo'' \cite{Ramos:2011}. 
%
Caso haja dúvida acerca de um desses requisitos, uma perícia técnica pode ser convocada.
%
Nesse caso, durante a perícia é analisado o autor da evidência, ou seja, verifica-se sua fonte e se a mesma não foi alterada no processo.


Em uma infraestrutura física, a coleta de evidências pode ser feita de forma relativamente simples, bastando-se remover o recurso físico, transportar este para um laboratório e lá analisar os dados. 
%
Para limitar a exposição da evidência a manipulações indevidas, ela pode ser mantida em uma sala-cofre, à qual o acesso é controlado.
%
A reprodutibilidade do processo de coleta e a manutenção da integridade da evidência são, então, tarefas bem diretas.


Já em um cenário de computação em nuvem, especialmente as de infraestrutura auto-escalável, existe um conjunto de novos desafios. 
%
\marcos{REMOVIDO: ``O recurso físico não pode mais ser removido pois o mesmo é utilizado por outros usuários não relacionados a investigação, fazê-lo constituiria violação de privacidade'' -- NOTA: ``o mesmo'' não deve ser usado para substituir ``ele''... a frase ``antes de entrar no elevador, verifique se o mesmo se encontra parado neste andar'' é um exemplo de uso \textbf{incorreto} de ``o mesmo'' (eu descobri isso há uns 2 meses atrás...). Por favor corrija no documento INTEIRO: o correto é usar ``ele'', ou então repetir a palavra em questão - Hamilton: Feito}. \marcos{No texto original tem vários erros incorretos de tempo verbal: você usa passado para se referir a forense em infraestrutura física e presente/futuro para infraestrutura em nuvem. Isso não faz sentido algum: nem tudo no mundo é nuvem, então qualquer coisa que você falar sobre infra física é *presente* - Hamilton: Feito. Acho que você já tinha corrigido algumas ocorrências}
Primeiramente, em contraste com infraestruturas tradicionais, o recurso físico em princípio não pode ser removido: como os recursos são utilizados por outros usuários não relacionados à investigação, fazê-lo constituiria violação de privacidade.
%
A volatilidade dos recursos também torna a verificação do seu autor um processo mais complexo, pois o recurso que gera certa evidência pode deixar de existir algum tempo depois de fazê-lo \cite{SimouCloudChlng:2014}.
%
A integridade da evidência também acaba sendo uma tarefa mais complexa, pois ela precisa ser coletada, transportada e armazenada, o que evidencia a necessidade de preservação da cadeia de custódia.
%
Infelizmente, a violação de qualquer uma dessas características pode colocar em dúvida a credibilidade da evidência.


\subsection{Volume de dados para coleta}
\label{sec:volumedados}

O processo de coleta de evidências na forense digital herda suas práticas da forense tradicional, na qual isola-se cena do crime e coletam-se as evidências presentes. 
%
Transportando esse método para a forense digital, introduz-se a realização da cópia bit a bit da informação que se deseja analisar.
%
No passado, com as soluções manipulando quantidades bem menores de memória, disco e tráfego, tal prática não era considerada muito problemática. 
%
Entretanto, nas atuais soluções, aplicações e arquiteturas em nuvem, o volume de dados é consideravelmente maior \cite{QuickIncreaseVolumeImpact:2014}.
%
%Por exemplo, em 2014 investigadores forenses tinham em média 6 meses de backlog para analisar \cite{QuickIncreaseVolumeImpact:2014} 
\marcos{Legal ter um exemplo, mas precisa *discutir* esse exemplo. E daí que tinha 6 meses de backlog? O que isso tem a ver com nuvem?! Vc não explicou o cenário, então pra mim como leitor esse backlog pode ser de um servidor a um tiozinho que anota em papel a entrada de pessoas... Mais CLAREZA!!! - Hamilton: Removido. Da forma com que é mencionado na referência, os 6 meses podem ser associados a processos pouco eficiêntes}.

Encontrar uma forma de armazenar menos informações, de modo a tornar a fase de análise mais rápida e eficiente, é um importante passo para permitir investigações forenses mais céleres.


\subsection{Privacidade e jurisdição}
\label{sec:violacaoprivacidadejuriscdicao}

No método tradicional de coleta de evidências para análise, isola-se o ambiente e as evidencias são removidas. \marcos{Eu acho que já li essa frase...}
%
Transportando para a forense digital, temos a prática de remover o equipamento para realização de cópia bit a bit da evidência. \marcos{Eu acho que já li essa frase... (sim, está repetitivo... resuma em uma frase curta, potencialmente fazendo referência à seção anterior)}
%
Nas soluções de infra estrutura física esta prática não traz grandes problemas, pois os objetos ou indivíduos sob investigação estão diretamente relacionados ao equipamento removido.
%
Nas soluções em nuvem, entretanto a adoção de tal prática não é recomendada porque o recurso físico é compartilhado por vários usuários, inclusive indivíduos não envolvidos na investigação, de modo que remover tais recursos configuraria violação de privacidade.
%
Como um complicador adicional, o fato de os dados não estarem armazenados no mesmo território em que a investigação é realizada acaba demandando acordos de cooperação jurídica entre as partes, o que nem sempre é possível \cite{SimouCloudChlng:2014}.


Neste cenário encontrar uma forma de coletar a evidência sem violar jurisdição e privacidade ganham importância.


\subsection{Coleta de evidências de memória volátil de máquinas em nuvem}
\label{sec:forensenuvem}

A prática de armazenar histórico de tráfego de rede e alterações de dados armazenados em disco já é bem difundida na comunidade forense\marcos{Qual comunidade? Do Jacarezinho? De Heliópolis? De novo, CLAREZA!!! - Hamilton: Feito}.
%
Por outro lado, a memória volátil de computadores não costuma receber o mesmo tratamento: suas alterações quase nunca são armazenadas, seja por questões de desempenho \marcos{NÃO use a palavra ``performance'' em textos científicos. Reserve o uso de ``performance'' para uma apresentação de música ou dança... em computação, o nome certo é ``desempenho'' - Hamilton: Falha minha, um momento de fraqueza} ou por simples praticidade, dado a alta sobrevida dessas informações.
%
Infelizmente, isso acaba dificultando a análise de uma classe específica de ataques, conhecidos como injeção de código em memória \cite{CaseMemoryForensics:2014}. 
%
Assim, quando usados contra uma arquitetura em nuvem, tais ataques não deixam rastros quando recursos de processamento virtuais são desativados e sua memória é liberada \cite{VomelMemoryAcquisition:2013,CaseMemoryForensics:2014}.
%
Em particular, têm especial interesse quatro tipos particulares dessa família de ameaças \cite{CaseMemoryForensics:2014}:


\begin{itemize}
 \item \textbf{Injeção remota de bibliotecas}: Um processo malicioso força o processo alvo a carregar uma biblioteca em seu espaço de memória.
 %
 Como resultado, o código da biblioteca carregada executa com os mesmos privilégios do executável em que ela foi injetada. 
 %
 Tal estratégia, comumente usada para instalar malwares, pode fazer com que uma biblioteca maliciosa armazenada no sistema seja distribuída por vários processos de uma mesma máquina, dificultando sua remoção \cite{MillerRemoteLibraryInjection:2004}.
 
 \item \textbf{Inline Hooking}: Um processo malicioso escreve código como uma sequência de bytes diretamente no espaço de memória de um processo alvo, e então força este último a executar o código injetado. 
 %
 O código pode ser, por exemplo, um \textit{script} de \textit{shell}.
 

 \item \textbf{Injeção reflexiva de biblioteca}: Um processo malicioso acessa diretamente a memória do processo alvo, inserindo nela o código de uma biblioteca na forma de uma sequência de bytes, e então força o processo a executar essa biblioteca. 
 %
 Nessa forma de ataque, a biblioteca maliciosa não existe fisicamente; isso torna tal estratégia de injeção de código potencialmente mais atrativa, pois o carregamento da biblioteca não é registrado no sistema operacional (SO), dificultando a detecção do ataque \cite{FewerReflectiveLibraryInject:2008}.
 
 \item \textbf{Injeção de processo vazio}: Um processo malicioso dispara uma instância de um processo legítimo no estado ``suspenso''; a área do executável é então liberada e realocada com código malicioso.
\end{itemize}





