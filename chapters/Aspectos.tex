% ----------------------------------------------------------
% Capitulo de Aspectos conceituais e economigos 
% ----------------------------------------------------------
\chapter{Aspectos conceituais e humanos}

\section{Nuvem computacional}
\label{sec:computacaonuvem}

Uma nuvem computacional é um modelo de infraestrutura no qual recursos compartilhados em quantidade configurável, acessíveis via rede, são alocados e desalocados com esforço mínimo de gerenciamento por parte de um provedor de serviços. \cite{NIST2011}
%
Há três modelos principais de comercialização de uso da nuvem \cite{NIST2011}: \textit{software} como serviço (\textit{Software as a Service} -- SaaS), na qual se provê o \textit{software} que será usado pelo cliente; plataforma como serviço (\textit{Platform as a Service} -- PaaS), na qual se provê o ambiente para que o cliente desenvolva, teste e execute seu \textit{software}; e, o tipo mais pertinente para este trabalho, Infraestrutura como serviço (\textit{Infrastructure as a Service} -- IaaS), na qual são fornecidos recursos computacionais básicos, como processamento e memória, em geral de forma virtualizada.
%
Uma arquitetura usada hoje nas soluções em núvem são as auto-escaláveis onde recursos são altamente voláteis com recursos sendo alocados e desalocados a qualquer momento. Estas implementações tem a vantagem de usar os recursos de nuvem de uma forma mais eficiênte e menos necessidade de intervenção humana.


\section{Conteinerização}
\label{sec:conteiner}

Segundo \cite{AmazonContainer}, conteiner é um método de virtualização do sistema operacional que permite executar uma aplicação e suas dependências em um processo com os recursos como disco, memória e rede isolados.
%
Diferentemente de máquinas virtuais a virtualização com contêineres é feita no nível do SO nativo, tem uma implementação mais simples eliminado camadas entre o aplicativo executado e o \textit{hardware} físico permitindo maior granularidade no controle sobre esses recursos melhorando a eficiência da infrestrutura.
%
Uma tecnologia bastante utilizada para esse propósito são Contêineres Linux (LXC) \cite{Linuxcontainers.org2015}, que aproveitam-se de funcionalidades como \textit{cgroups}, \textit{kernel namespacing} e \textit{chroot} do kernel do Linux para auxiliar no gerenciamento e isolamento de recursos virtuais.
%


\section{Forense digital e seus desafios atuais}
\label{sec:forensedigital}

Forense digital (tambem conhecida por forense computacional) é um conjunto de técnicas de coleta e análise de interação entre humanos e computadores de forma que esta seja aceita em um processo legal.
%
Tal como a forense tradicional, a forense digital se baseia no princípio de Locard. Definido pelo médico francês Edmond Locard, o princípio de Locard estabelece que ``Quando um indivíduo entra em contato com outro objeto ou indivíduo, este sempre deixa vestígio deste contato''. \cite{Ramos:2011}
%
A forense digital quando aplicada a investigação de incidentes em soluções na nuvem enfrenta desafios adicionais relacionados a coleta, transporte e análise da evidência.
%
A primeira delas diz respeito a aceitabilidade de uma evidência. Para que uma evidência seja aceita em um processo legal é necessária que a cadeia de custódia relacionada a evidência tenha sido garantida.
%
Cadeia de custódia é o processo de documentação da história cronológica da evidência de modo a saber onde a evidência esteve e quem teve acesso a ela (referencia). O SENASP 2013 diz que: ``CADEIA DE CUSTÓDIA: sistemática de procedimentos que visa à preservação do valor probatório da prova pericial caracterizada.''
%
Assim a Forense digital tem por objetivo a investigação de evidências digitais da interação entre homem e máquina de modo a reconstruir a cadeia de evêntos no passado de forma que suas conclusões sejam validadas por terceiros.


\subsection{Aceitabilidade da evidência em processo legal.}
\label{sec:credibilidadeaceitabilidadeevidencia}

O processo de análise forense no evento de um crime digital é descrito no EDIPM - \textit{Enhanced Digital Investigation Process Model} por 4 fases: Identificar, preservar, examinar, apresentar. \cite{GrisposChallengesCloudComputing:2012}
%
Na fase de preservação da evidência deve ser feita de forma que os autores descrevem como ``forensicamente aceitável'' isto é, coletar as evidências de forma que as mesmas sejam aceitas em um processo legal e não sejam invalidadas durante o processo.
%
Segundo \cite{Ramos:2011} a aceitabilidade de uma evidência digital em um processo legal deve atender aos seguintes requisitos: Autenticidade Processo pelo qual se pode garantir a autoria do documento eletrônico, ou seja, não permite dúvida quanto à identificação do autor.
%
e Integridade: Permite atestar a “inteireza do documento eletrônico após sua transmissão, bem como apontar eventual alteração irregular de seu conteúdo”.
%
Caso haja dúvida sob qualquer um dos requisitos uma perícia técnica pode ser convocada, nesta será análisada o autor da evidência ou seja sua fonte e se a mesma não foi alterada no processo.
%
Em infra-estrutura física esta coleta era relativamente simples, bastava-se remover o recurso físico, transporta-lo para um laboratório e lá análisar a evidência. A Evidência era mantida em uma sala-cofre onde o acesso era controlado.
%
A reprodutibilidade do processo de coleta e a manutenção da integridade da evidência eram tarefas bem diretas.
%
A computação em núvem, especialmente as de infraestrutura auto-escalável troxeram um conjunto de desafios para se atingir este requisito. O recurso não pode mais ser removido pois o mesmo é utilizado por outros usuários não relacionados a investigação, fazê-lo constituiria violação de privacidade.
%
A volatilidade dos recursos tornou a verificação do seu autor um processo mais complexo pois o recurso que a gerou pode não existir mais.
%
A integridade da evidência também tornou-se mais complexa pois ela precisa ser coletada, transportada e armazenada. O processo de cadeia de custôdia ganhou grande visibilidade neste quesito.
%
Violação de qualquer uma das caracteristicas citadas anteriormente põe em dúvida a credibilidade da evidência.

\subsection{Volume de dados para coleta}
\label{sec:volumedados}

O processo de coleta da evidência na forense digital herdou suas práticas da forense tradicional onde isola-se cena do crime e coletam-se as evidências presentes. 
%
Transportando para a forense digital criou-se o hábito de se realizar cópia bit a bit da informação que se deseja investigar. 
%
No passado, com as soluções tendo bem menos capacidade de memória, disco e tráfego, tal prática não trazia problemas. Nas atuais soluções, aplicações e arquiteturas em núvem o volume de dados aumentou  muito.
%
Em (encontrar a data) investigadores forenses tinham em média 6 meses de backlog para analisar. Em conversas informais com analistas forenses é comum a métrica de em média apenas 2\% do material coletado ser útil a análise.
%
Encontrar uma forma de armazenar menos informações de modo a tornar a fase de análise mais rápida e eficiênte ajudará nas investigações.

\subsection{Privacidade e jurisdição}
\label{sec:violacaoprivacidadejuriscdicao}

Na metodologia tradicional de coleta de evidências para análise isola-se o ambiente e as evidencias são removidas. Transportado para a forense digital temos a prática de remover o equipamento para realização de cópia bit a bit da evidência.
%
Nas soluções de infra estrutura física esta prática não trás problemas, os objeto ou indivíduos sob investigação estão diretamente relacionados ao equipamento removido. 
%
Nas soluções em nuvem esta prática não pode mais ser utilizada pois como o recurso físico é compartilhado por vários usuários não envolvidos na investigação, remove-los configura violação de privacidade.
%
Um complicador a mais é o fato de os dados não estarem armazenados no mesmo território em que a investigação é realizada demandando acordos de cooperação jurídica entre as partes o que nem sempre é possível.
%
Neste cenário encontrar uma forma de coletar a evidência sem violar jurisdição e privacidade são de grande importância num futuro próximo.

\subsection{Coleta de evidências de memória volátil de máquinas em nuvem}
\label{sec:forensenuvem}

Na metodologia tradicional de coleta de evidências para análise isola-se o ambiente e as evidencias são removidas. Transportado para a forense digital temos a prática de remover o equipamento para realização de cópia bit a bit da evidência.
%
Nas soluções de infra estrutura física esta prática não trás problemas, os objeto ou indivíduos sob investigação estão diretamente relacionados ao equipamento removido. 
%
Nas soluções em nuvem esta prática não pode mais ser utilizada pois como o recurso físico é compartilhado por vários usuários não envolvidos na investigação, remove-los configura violação de privacidade.
%
Um complicador a mais é o fato de os dados não estarem armazenados no mesmo território em que a investigação é realizada demandando acordos de cooperação jurídica entre as partes o que nem sempre é possível.
%
Neste cenário encontrar uma forma de coletar a evidência sem violar jurisdição e privacidade são de grande importância num futuro próximo.






