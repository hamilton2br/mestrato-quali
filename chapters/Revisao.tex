% ----------------------------------------------------------
% Revisão bibliográfica
% ----------------------------------------------------------
\chapter{Revisão da literatura}
\label{chp:revisão}

Os principais aspectos relacionados ao tratamento de evidências para análise forense em nuvem são: coleta, transporte, armazenamento, garantia da cadeia de custódia e reprodutibilidade do processo de coleta. 
%
Este capítulo descreve o estado da arte e discute como a presente proposta se insere nesse contexto.


\section{Soluções para análise forense em nuvem}
\label{sec:estadodaarte}

%\marcos{Do jeito que você está apresentando as figuras, você pode ser criticado pelo fato delas terem sempre muito mais informação do que você discute no texto. Não é crítico, mas talvez (se quiser se precaver de críticas) você queira repensar a forma como você discute a solução, para cobrir de forma mais ampla as caixas mostradas na figura (não precisa ser todas as caixinhas, mas pelo menos algumas é bom o leitor saber pra que servem... aí é analisar uma a uma o que potencialmente poderia ser melhor descrito. - Hamilton: Feito}

Nesta seção são descritas algumas das principais propostas existentes para coleta de evidências forenses, fornecendo uma visão geral do estado da arte na área.

\subsection{Modelo automático de aquisição de dados para fins forenses}
\label{sec:aquisicaoautomatica}

%\marcos{é sempre ``a Figura X'' (maiúsculo, porque é nome próprio), e ``essa figura'' (minúscula, porque não é nome próprio'' - Hamilton : Feito}

O modelo proposto por \cite{ReichertAutoAcquisition:2015}, ilustrado na Figura \ref{fig:ReichertAutoAcquisitionModel}, é um processo de coleta de evidências integrado ao \textit{hipervisor} e disparado por algum sistema de detecção de intrusão. 
%
A partir do momento em que uma ameaça é detectada, o modelo dita que sejam criados instantâneos (\textit{snapshots}) das máquinas virtuais comprometidas. 
%
O intervalo de tempo em que os instantâneos de memória são gerados é configurável, e todos eles são armazenados em persistência.
%
O modelo se preocupa em, de forma automatizada, excluir informações de clientes não relacionados à investigação e também em armazenar o restante em local seguro.
%
A proposta não descreve os detalhes do armazenamento, embora afirma que isso deve ser feito de forma forensicamente aceitável.


Para agregar e analisar as evidências coletadas, o modelo faz uso do Resposta Rápida Google (\textit{Google Rapid Response} -- GRR), uma ferramenta de resposta a incidentes criado no projeto 20\% na Google para facilitar análises forenses de forma remota (e.g., acesso de baixo nível ao armazenamento e à memória das máquinas analisadas) \cite{GRRRapidResponse:2013}.
%
O tratamento de evidências usa um motor de regras baseado em um conjunto de descrições de ameaças conhecidas, as quais são armazenadas em um banco de dados.
%
Caso alguma evidência coletada coincida com ameaças armazenadas, esse motor alerta um usuário humano para uma avaliação mais detalhada \cite{ReichertAutoAcquisition:2015}.
%
A característica deste modelo que mais contribui para a forense digital é a automação do processo de coleta, já que, ao menos em parte, ele dispensa intervenção humana. 


%Quando acionado, o Motor gera instantâneos das máquinas virtuais administradas por \textit{Hipervisor 1} e \textit{Hipervisor 2} e os envia para o Armazenamento de \textit{Back-End}.
%
%Os instantâneos são instalados em um ambiente controlado junto com agentes GRR que extraem os dados para análise. 
%
%Estes dados são então comparados com ameaças já conhecidas pelo modelo.

\begin{figure}[htb!]
\footnotesize
\caption{Modelo Automático de Aquisição de Dados Forenses} %\marcos{A fonte do texto está muito pequena... Colocando em 100\% de zoom no pdf, a menor fonte que pode aparecer na figura tem que ser maior ou igual a fonte do footnote do documento. Fontes menores do que isso precisam ser aumentadas. \textbf{Isso vale para todas as figuras.} Além disso: por que ``Actor'' e não ``Ator''...? - Hamilton: Feito}}
\includegraphics[scale=0.70]{ReichertAutoAcquisitionModel.pdf}
\centering
\label{fig:ReichertAutoAcquisitionModel}
\begin{center}
Adaptado de \cite{ReichertAutoAcquisition:2015} 
\end{center}
\end{figure}


\subsection{Introspecção em máquina virtual}
\label{sec:VMI}

A proposta de \cite{PoiselVMI:2013} é baseada na técnica de Introspecção em Máquina Virtual (\textit{Virtual Machine Introspection} -- VMI) para coleta de memória volátil. 
%
Essa técnica se apoia no fato de que o \textit{hipervisor} mapeia os recursos alocados para máquinas virtuais nos recursos físicos correspondentes da máquina hospedeira.
%
Este mapeamento é usado para permitir que a memória volátil copiada da máquina virtual seja reconstruída em uma máquina física, para análise posterior.
%
A proposta realiza coleta contínua de instantâneos de memória durante o funcionamento do sistema, sem distinção do que aconteceu antes ou depois do fato de interesse, e todos os instantâneos de memória são armazenados para análise.
%
Visando eliminar a chance de inconsistências no instantâneo de memória volátil, a máquina virtual tem sua execução suspensa durante o processo de extração.


Em \cite{PoiselVMI:2013}, no capítulo 3.1, o próprio autor menciona que a necessidade de tradução de endereços de memória da máquina virtual em endereços de memória da máquina física hospedeira dificulta a utilização da técnica em larga escala.
%
Como essa tradução depende de conhecimento do que está sendo executado na máquina virtual, uma solução baseada em VMI não é completamente portável, sendo necessárias adequações para diferentes clientes.
%
Além disso, esta tradução de endereços pode ser computacionalmente custosa.%\marcos{É isso mesmo? Não ficou claro na frase original o que era ``computacionalmente custoso'', então assumi aqui que é ``a tradução''. - Hamilton: sim era mas você tem razão, mudei}.
%

\subsection{Virtuoso}
\label{sec:virtuoso}

Também na vertente de introspecção de máquina virtual, \cite{Dolan-GavittSemanticGap:2011} propõe o \textit{Virtuoso}, um arcabouço de coleta de informações de processos específicos em uma máquina virtual.
%
O arcabouço funciona em três fases.
%
A primeira realiza um estudo em uma máquina virtual de testes, mapeando o conjunto de instruções executado pelo processo do qual se deseja coletar dados de memória. 
%
O estudo é realizado no Ambientes de Treino e o Leitor de Instruções coleta e armazena as instruções geradas pelo processo alvo da análise e os armazena no banco de dados de Instruções Armazenadas.
%
Na segunda fase, o Analisador de Instruções cria um executável a partir do conjunto de instruções coletado na fase anterior. Estas instruções precisam ter suas referências de memória traduzidas para que seja possível executá-las fora do ambiente original, o que ocorre no Tradutor de Instruções.
%
Com o executável, a terceira fase usa um Ambiente Virtual Seguro na máquina hospedeira capaz de acessar os endereços de memória do Ambiente Virtual Não Confiável, tornando possível coletar instantâneos de memória do processo em execução.
%
A Figura \ref{fig:Dolan-GavittSemanticGap} ilustra o funcionamento desse arcabouço. 


A característica deste modelo que mais contribui para a forense digital é a capacidade de coletar instantâneos de memória de um processo específico e reproduzí-lo em um ambiente confiável. Entretanto  sua atuação se dá apenas após o evento que se deseja avaliar forensicamente.
%


\begin{figure}[htb!]
\footnotesize
\caption{Virtuoso}
\includegraphics[scale=0.70]{Dolan-GavittSemanticGap.pdf}
\centering
\label{fig:Dolan-GavittSemanticGap}
\begin{center}
Adaptado de \cite{Dolan-GavittSemanticGap:2011} 
\end{center}
\end{figure}


\subsection{Abordagem baseada em logs}
\label{sec:modelologs}

O arcabouço proposto por \cite{SangLogApproach:2013} é um sistema que funciona em parceria com o provedor de nuvem: o provedor último envia informações ao arcabouço, que por sua vez as armazena em um local adequado, de forma centralizada.
%
O conjunto de informações armazenadas é negociado antecipadamente com o provedor de nuvem, indo desde instantâneos de memória volátil até pacotes trafegados nas interfaces de rede da máquina virtual.
%
O arcabouço coleta informações continuamente e usa cálculo de hash das evidências enviadas pelo provedor de nuvem para garantir que elas não foram alteradas durante o transporte.
%
A Figura \ref{fig:SangLogApproach} ilustra o funcionamento da solução, focando em um caso específico de log de rede, de modo similar ao descrito em \cite{SangLogApproach:2013}.


Assim como as propostas anteriores, o arcabouço em questão também não faz distinção do que aconteceu antes ou depois do fato de interesse, mas  coleta constantemente informações da máquina virtual.
%
Outra potencial limitação é que o arcabouço depende da cooperação do provedor de nuvem. 
%
Tal dependência é uma estratégia considerada fraca pela comunidade forense  \cite{ClarkeReviewOfChallenges2015}, pois a prioridade do Provedor de Serviços de Nuvem (\textit{Cloud Service Provider} -- CSP) é a de garantir a disponibilidade do serviço, não de coletar evidências.
%

\begin{figure}[htb!]
\footnotesize
\caption{\textit{A Log Based Approach Model}}
\includegraphics[scale=0.80]{SangLogApproach.pdf}
\centering
\label{fig:SangLogApproach}
\begin{center}
Adaptado de \cite{SangLogApproach:2013} 
\end{center}
\end{figure}


\subsection{Abordagem baseada em backups}
\label{sec:modelobackup}

O trabalho descrito em \cite{DezfouliBackupApproach:2012} é voltado a dispositivos móveis e tem como principal característica a preocupação com as limitações de armazenamento do dispositivo.
%
Por essa razão, o processo de coleta de instantâneos de memória volátil e armazena separadamente as informações de cada processo que está ativo de modo a permitir um gerenciamento adequado do espaço de armazenamento. 
 %
%\marcos{O que seria ``armazenar por processo ativo''? Eu nem sabia que dava para ``armazenar por processo passivo''... - Hamilton: Melhorei}.
%
A solução também se preocupa em descartar informações de processos que foram terminados e removidos da memória, além de buscar o uso consciente do espaço de armazenamento disponível no dispositivo.
%
A Figura \ref{fig:DezfouliBackupApproach} mostra, em alto nível, a forma como o armazenamento de evidências é gerenciado.

Como em diversas outras propostas, processo de coleta de informações de memória volátil em \cite{DezfouliBackupApproach:2012} é executado continuamente, independente de eventos de interesse (e.g., detecção de ameaças). 
%
Um outro fator que conta como desvantagem nesta proposta é o processo não armazenar histórico das coletas anteriores, como é possível ver na Figura \ref{fig:DezfouliBackupApproach}. A coleta anterior é substituída pela nova.
%


\begin{figure}[htb!]
\footnotesize
\caption{\textit{Backup Approach Model}}
\includegraphics[scale=0.80]{DezfouliBackupApproach.pdf}
\centering
\label{fig:DezfouliBackupApproach}
\begin{center}
Adaptado de \cite{DezfouliBackupApproach:2012} 
\end{center}
\end{figure}


\subsection{Arcabouço forense para OpenStack}
\label{sec:frost}

As Ferramentas Forenses para Arcabouço OpenStack (\textit{FoRensic OpenStack Tools} -- FROST), proposta por \cite{DykstraFROST:2013}, consiste em um conjunto de bibliotecas integradas ao OpenStack, um dos arcabouços de gerenciamento de infraestruturas virtualizadas bastante difundido \cite{StackFramework:2018}.
%
Por meio dessa integração, o FROST expõe um conjunto de APIs que podem ser usadas por aplicações de coleta de evidências forenses.
%
Essas APIs dão acesso a recursos da máquina virtual administrada, tais como disco, \textit{logs} de tráfego de rede e memória volátil.
%
A proposta descreve apenas o arcabouço, deixando a critério do usuário detalhes como periodicidade e tamanho da coleta, bem como a forma de transporte da evidência e onde ela é armazenada.

\begin{figure}[htb!]
\footnotesize
\caption{\textit{FoRensic OpenStack Tools}}
\includegraphics[scale=0.80]{DykstraFROST.pdf}
\centering
\label{fig:DykstraFROST}
\begin{center}
Adaptado de \cite{DykstraFROST:2013} 
\end{center}
\end{figure}
%

A Figura \ref{fig:DykstraFROST} ilustra a integração entre FROST e o arcabouço OpenStack.
%
Nela é possível ver os dois pontos desta integração. O primeiro em sua camada de interface de administração web, através da API de Computação (\textit{Compute API}) onde o usuário pode acionar a coleta de artefatos para análise.
%
A segunda ocorre em seu núcleo, adiciona novas chamadas a API Nova (\textit{Nova API}) do OpenStack para viabilizar a coleta de informações da máquina virtual e também integra com os processos de rede para extrair logs de tráfego da rede.


O autor declara que FROST segue as práticas definidas no Grupo de Pesquisa Cientifica Em Evidência Digital (\textit{Scientific Working Group on Digital Evidence} -- SWGDE) e do Manual de Busca e Apreensão do Departamento de Justiça Norte-Americano \cite{DykstraFROST:2013}.
%
De todas as propostas avaliadas, a FROST é a única que mostra preocupação com adequação a questões legais como cadeia de custódia e integridade da evidência.
%

\subsection{Forense como serviço}
\label{sec:frost}

O trabalho descrito em \cite{GeorgeDF2CE:2012} se concentra em monitoramento de rede, operando em uma arquitetura de Forense Como Serviço (\textit{Forensics as a Service} -- FaaS). 
%
Conforme ilustrado na Figura \ref{fig:GeorgeDF2CE}, a arquitetura da solução consiste em um conjunto de ferramentas com capacidade de descobrir automaticamente as interfaces sob monitoramento, além de coletar evidências de tais máquinas e armazená-las.


O processo de autodescoberta e associação das evidências com usuários de rede é realizado por um motor baseado em ontologias armazenadas em um banco de dados próprio. 
%
As ontologias são utilizadas para determinar a associação entre as evidências coletadas e os usuários ou artefatos que as geraram.
%
O Minerador de dados implementa um algoritmo de descoberta de arquivos de \textit{log} de artefatos de rede definidos como relevantes.
%
O Monitor de rede é responsável por interceptar o tráfego entre as partes consideradas suspeitas e por fim, a Monitoração do Sistema em Tempo Real captura informações específicas do sistema em execução como instantâneos dos processos sob investigação.
%
Entretanto, a proposta se concentra apenas no processo de coleta, enquanto a descrição dos mecanismos de armazenamento e transporte não é detalhada.


\begin{figure}[htb!]
\footnotesize
\caption{\textit{Digital Forensic Framework for Cloud Environment}}
\includegraphics[scale=0.70]{GeorgeDF2CE.pdf}
\centering
\label{fig:GeorgeDF2CE}
\begin{center}
Adaptado de \cite{GeorgeDF2CE:2012} 
\end{center}
\end{figure}


\subsection{Abordagem de indexação de dados coletados para fins forenses}
\label{sec:indexacaoforense}

Na mesma vertente da solução de forense como serviço descrita na Subseção \ref{sec:frost}, \cite{FaaSIndexedSearch:2012} trata o problema de grande volume de dados coletados por meio de um serviço de coleta e indexação de evidências.
%
O serviço espera receber dados da execução do comando unix DD \cite{UnixManPagesDD} nas máquinas alvo, nas quais, apoiado em processos de Extrair, Transformar e Carregar (\textit{Extract, Transform and Load} -- ETL) e MapReduce \cite{MapReduce:2008}, os dados são disponibilizados para consulta pelos investigadores.
%
A coleta ocorre continuamente, em intervalos de tempo configuráveis.


A Figura \ref{fig:FaaSIndexedSearch} mostra a arquitetura da solução. Um Armazenamento Acessado via Rede (\textit{Network Accessed Storage} -- NAS) é usado para armazenar as evidências coletadas.
%
Antes da análise dos dados pelos processos de MapReduce é necessário executar o processo de ETL para adequar a informação a necessidade específica da investigação. Esta fase ocorre nos Nós Filtrando.
%
Em seguida a informação é submetida ao processo de MapReduce e armazenado em um HBase \cite{Hbase2018}.
%
O Nó Mestre tem o papel de orquestrador enviando os comandos de indexação dos dados coletados e recebendo as requisições de busca dos usuários via seu Servidor Web.


Embora interessante, a solução não deixa claro a localização do armazenamento dos dados coletados, nem quem é responsável pela infraestrutura de armazenamento.
%
Também não é discutido como os dados são transportados até o ponto de armazenamento, nem como garantir que esses dados não sejam alteramos no processo.
%


\begin{figure}[htb!]
\footnotesize
\caption{\textit{Digital Forensic as a Service - Indexed Data}}
\includegraphics[scale=0.60]{FaaSIndexedSearch.pdf}
\centering
\label{fig:FaaSIndexedSearch}
\begin{center}
Adaptado de \cite{FaaSIndexedSearch:2012} 
\end{center}
\end{figure}


\section{Aspectos relacionados a coleta de evidência}
\label{sec:coletadeevidencia}

Para uma discussão mais estruturada, nas próximas subseções os trabalhos mencionados na Subseção \ref{sec:VMI} são agrupados e avaliados com base nos diferentes aspectos que abordam.

\subsection{Acessar e coletar as informações de memória das máquinas virtuais em nuvem}
\label{sec:coletadeevidencia}

Diversos trabalhos de análise forense na nuvem se concentram na coleta de dados ``após o fato'', ou seja, após a intrusão ser detectada \cite{ReichertAutoAcquisition:2015,PoiselVMI:2013,DykstraFROST:2013,GeorgeDF2CE:2012,SangLogApproach:2013}. 
%
Os processos de coleta descritos nesses trabalhos podem ser iniciados de forma manual ou automaticamente, via integração com um mecanismo de detecção de intrusão. 
%
No caso específico de memória volátil, tal forma de coleta não consegue descrever como era a memória antes da intrusão, pois o processo só é acionado depois da detecção do ataque. 
%
%A capacidade de saber como era a memória antes do fato é descrita por \cite{Case_Memory_Forensics:2014} como necessária para viabilizar a abordagem de coletar o suficiente para realizar a investigação pois permite comparar dois instantâneos de memória e minimizar o volume coletado antes do fato. 
Tal limitação pode trazer prejuízos à investigação, dado que algumas análises dependem exatamente da capacidade de se comparar dois momentos da memória \cite{CaseMemoryForensics:2014}. 
%
Entre os trabalhos estudados, a única proposta encontrada que leva tal necessidade em consideração é \cite{DezfouliBackupApproach:2012}, que propõe que o dado seja armazenado no próprio equipamento sob análise.
%
%Infelizmente, entretanto, essa abordagem não é aplicável ao cenário em nuvem, pois leva a perda de informações importantes caso a máquina virtual seja despejada e seus recursos liberados.
Infelizmente, entretanto, a aplicação de tal abordagem no cenário em nuvem é pouco viável, pois pode levar à perda de informações importantes caso a máquina virtual ou contêiner seja desativada, tendo seus recursos liberados.
%

%Ainda na coleta de informações, os autores \cite{Reichert_Auto_acquisition:2015} e \cite{George_DF2CE:2012} sugerem a abordagem de forense ao vivo onde os dados são constantemente coletados sem distinção do antes ou depois do fato. 
Existem ainda trabalhos voltados à coleta de informações durante a execução do sistema, nos quais os dados são constantemente coletados sem distinção do que aconteceu antes ou depois do fato de interesse.
%
Esse é o caso de trabalhos como \cite{PoiselVMI:2013,DykstraFROST:2013,SangLogApproach:2013,Dolan-GavittSemanticGap:2011}, que adotam a estratégia de isolar e parar a máquina virtual para em seguida realizar o processo de coleta. 
%
%Nas duas estratégias citadas anteriormente, o problema do grande volume de informações coletadas não é abordado pelo autores nem o cenário onde é necessário coletar evidências de uma máquina virtual que já foi despejada do pool e os recursos liberados. 
Embora interessantes, as abordagens descritas nesses trabalhos podem levar a um elevado volume de dados coletados.
%
Além disso, elas não tratam o cenário em que é necessário coletar evidências quando são liberados os recursos virtuais que as contêm.


\subsection{Capacidade de reproduzir o processo e obter os mesmos resultados}
\label{sec:reprodutibilidade}

Se, durante uma análise forense, analistas diferentes obtêm resultados distintos ao executar o mesmo procedimento de coleta, a evidência gerada não tem credibilidade, inviabilizando seu uso em um processo legal. 
%
Por essa razão, a reprodutibilidade do processo de coleta é uma parte importante da geração de evidências para análise forense.
%
Infelizmente, entretanto, nenhuma das propostas encontradas na literatura atualmente permite tal reprodutibilidade em cenários de nuvem, em que máquinas virtuais ou contêineres são desativados e seus recursos físicos liberados.
%
Afinal, todas elas dependem da existência do recurso virtual para a repetição do processo de coleta.

\subsection{Não violar privacidade ou jurisdição das partes não envolvidas na investigação}
\label{sec:legais}

Em um ambiente de nuvem pública, remover o \textit{hardware} para análise posterior pode levar à violação de privacidade de usuários pessoa física e jurídica.
%
A razão é que o multi-inquilinato desse cenário faz com que uma mesma máquina física guarde informações de diversos clientes, alguns dos quais podem não estar envolvidos na investigação em curso.
%
Diversos trabalhos na literatura tratam esse problema adequadamente, por meio de duas estratégias principais: a primeira, adotada em \cite{ReichertAutoAcquisition:2015,GeorgeDF2CE:2012,PoiselVMI:2013,DykstraFROST:2013,FaaSIndexedSearch:2012}, consiste em coletar dados pertinentes à investigação e armazená-los fora da nuvem; a segunda, empregada em \cite{SangLogApproach:2013} e que constitui um caso específico de \cite{GeorgeDF2CE:2012}, depende da cooperação do provedor de serviços de nuvem para conseguir as informações necessárias à investigação. 
%
Depender do provedor de serviços de nuvem é uma estratégia pouco recomendada  \cite{ClarkeReviewOfChallenges2015}, entretanto, pois (1) o volume de dados de usuários pode forçar os provedores a limitar o tamanho dos \textit{logs} armazenados, e (2) caso ocorra uma indisponibilidade causada por um ataque, o objetivo do provedor será o de restabelecer o serviço, não necessariamente o de preservar evidências. 


\subsection{Garantir a cadeia de custódia da evidência}
\label{sec:cadeiadecustodia}

Dentre os trabalhos analisados, apenas \cite{SangLogApproach:2013} aborda a questão da garantia da cadeia de custódia. 
%
Especificamente, o trabalho emprega \textit{hashes} para verificar a integridade da evidência, permitindo a detecção de alterações.
%
Uma limitação desse trabalho, entretanto, é que ele não deixa explícitos os mecanismos que poderiam ser utilizados para impedir acesso não autorizado (e, assim, potencial alteração) aos próprios \textit{hashes}. 
%
As propostas dos outros autores concentram-se apenas no aspecto técnico da coleta, sem discutir detalhadamente garantia de custódia.
%
Em geral, os trabalhos apenas mencionam que as evidências devem ser coletadas de forma forensicamente aceitável.

\section{Resumo}
\label{sec:resumo}

A Tabela \ref{tab:related-work} mostra um comparativo das soluções estudadas, considerando os aspectos discutidos nesta seção, posicionando as contribuições da proposta apresentada neste trabalho.

\begin{table}[htb!]
\footnotesize
\renewcommand{\arraystretch}{1.4}
\renewcommand{\tabcolsep}{0.5mm}
\centering
\caption{Comparativo de soluções de coleta de informações de memória de máquinas em nuvem para análise forense}
\label{tab:related-work}
\begin{tabular}{p{5.3cm}|L|L|L|L|L|L|L|L|L|}
\textbf{}						& \rot{\fancyname~(esta proposta)} 			& \rot{\cite{GeorgeDF2CE:2012}} 
							& \rot{\cite{PoiselVMI:2013}} 				& \rot{\cite{DykstraFROST:2013}}
							& \rot{\cite{FaaSIndexedSearch:2012}} 			& \rot{\cite{ReichertAutoAcquisition:2015}}	
							& \rot{\cite{SangLogApproach:2013}} 			& \rot{\cite{Dolan-GavittSemanticGap:2011}} 
							& \rot{\cite{DezfouliBackupApproach:2012}} 				
\\ \hline
\textbf{Coleta evidência continuamente.}		& \cfig	& \xfig & \xfig & \xfig & \cfig & \xfig & \cfig & \xfig & \cfig  \\
\textbf{O processo é reprodutível.}		& \cfig	& \xfig & \xfig & \xfig & \xfig & \xfig & \xfig & \xfig & \xfig  \\
\textbf{Garante cadeia de custódia.}			& \cfig	& \xfig & \xfig & \xfig & \xfig & \cfig & \cfig & \xfig & \xfig  \\
\textbf{Preserva privacidade e jurisdição.} 		& \cfig	& \cfig	& \cfig	& \cfig	& \cfig	& \cfig	& \cfig	& \cfig	& \cfig	 \\
\end{tabular}
\end{table}
