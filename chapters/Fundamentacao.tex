% ----------------------------------------------------------
% Capitulo de Fundamentação teórica 
% ----------------------------------------------------------
\chapter{Fundamentação Teórica}
\label{chp:fundamentação}


Desde 2001, diversos modelos para a condução de investigação digital foram propostos. 
%
No \textit{First Digital Forensic Research Workshop} foi definido o primeiro modelo de processo investigativo genérico para atender às investigações envolvendo sistemas digitais.
%
Em seguida foi proposto o \textit{Abstract Digital Forensic Model}, mais detalhado que o modelo genérico anterior e o \textit{Integrated Digital Investigation Process}, baseado em teoria criminal e investigações não digitais do passado.
%
Finalmente em 2003, Carrier e Stafford propõem a última iteração na evolução do processo forense digital, o \textit{Enhanced Digital Investigation Process Model} \cite{SimouCloudChlng:2014}.
%
Entretanto, como tais modelos de investigação foram desenvolvidos antes da aparição de tecnologias de computação em nuvem, muitos partem do pressuposto que o investigador tem acesso e controle sobre o sistema sob investigação \cite{GrisposChallengesCloudComputing:2012}.
%
Novas pesquisas são necessárias para que a forense digital aborde apropriadamente as soluções em nuvem e suas particularidades.


Neste capítulo, são discutidos os principais conceitos que permeiam esse cenário.
%
Especificamente, após uma discussão geral sobre computação em nuvem e suas particularidades, são discutidas os características esperadas de um processo de forense digital robusto e sua interseção com gestão de incidentes.


\section{Nuvens computacionais e contêineres}
\label{sec:computacaonuvem}

Uma nuvem computacional é um modelo de infraestrutura no qual recursos compartilhados em quantidade configurável, acessíveis via rede, são alocados e desalocados com esforço mínimo de gerenciamento por parte de um provedor de serviços.
%
Existem três modelos principais de comercialização de uso da nuvem \cite{NIST2011}: 

\begin{itemize}
	\item \textit{Software} como serviço ( \textit{Software as a Service} -- SaaS ): No qual se provê o \textit{software} que será utilizado; nesse caso, os clientes do serviço são os usuários finais do software.
	
	\item Plataforma como serviço ( \textit{Platform as a Service} -- PaaS ): No qual se provê o ambiente para o desenvolvimento, teste e execução do \textit{software}; nesse caso, os clientes do serviço são desenvolvedores de aplicações.
	
	\item Infraestrutura como serviço ( \textit{Infrastructure as a Service} -- IaaS ): No qual são fornecidos recursos computacionais básicos, como processamento, memória e redes, em geral de forma virtualizada.
\end{itemize}

O tipo de serviço de nuvem mais pertinente para este trabalho é o IaaS, uma solução muito usada atualmente pela sua capacidade de prover recursos sob demanda de forma auto-escalável.
%
Nesse cenário, o uso intenso de tecnologias de virtualização costuma levar a recursos altamente voláteis, que são alocados e desalocados a qualquer momento pelo orquestrador da nuvem para suprir eventuais aumentos e reduções de demanda.
%
É possível até mesmo construir \textit{scripts} para a automatizar a construção da arquitetura desejada, permitindo a instanciação e interconexão de máquinas adequadas para a atividade fim do sistema.
%
Esses \textit{scripts}, escritos em linguagem específica como Puppet \cite{Puppet2018}, Chef \cite{Chef2018} ou Vagrant \cite{Vagrant2018}, podem conter instruções de como distribuir o tráfego de rede entre as diferentes instâncias de computação ou armazenamento.
%
O modelo IaaS também permite controle direto nos elementos de infraestrutura, que veremos ao longo deste documento ser uma característica que ajuda no entendimento da solução proposta.
%

Nos modelos SaaS e PaaS, o usuário não controla nem gerencia os elementos mais basicos de infraestrutura como armazenamento, sistema operacional e rede dos componentes de software oferecidos \cite{NIST2011}. Tal característica pode enviesar os resultados deste trabalho para a realidade de um provedor de nuvem específico.
%Dentre as vantagens de sua utilização, podem ser citadas a capacidade de usar os recursos de nuvem de uma forma mais eficiente, levar a uma menor necessidade de intervenção humana, e prover maior resiliência a variações de demanda do sistema.


%\section{Uso de contêineres}
%\label{sec:conteiner}

Uma tecnologia de virtualização possível para cenários de nuvem, e cuja utilização vem crescendo nos últimos anos, são os chamados contêineres \cite{containers-tech:2014}. 
%
Basicamente, um contêiner é um método de virtualização do sistema operacional que permite executar uma aplicação, bem como suas dependências, em um processo no qual recursos como disco, memória e rede permanecem isolados.
%
Diferente das máquinas virtuais, a virtualização com contêineres é feita no nível do sistema operacional (SO) nativo.
%
Como resultado, tem-se uma implementação de virtualização na qual eliminam-se camadas entre o aplicativo executado e o \textit{hardware} físico, permitindo maior granularidade no controle sobre esses recursos e melhorando a eficiência da infraestrutura.


Uma implementação bastante utilizada para esse propósito são os Contêineres Linux ( \textit{LinuX Conteiners} -- LXC ) \cite{Linuxcontainers.org2015}, que aproveitam-se de funcionalidades como \textit{cgroups}, \textit{kernel namespacing} e \textit{chroot} do núcleo do Linux para auxiliar no gerenciamento e isolamento de recursos virtuais.
%
Mais precisamente, a funcionalidade de \textit{cgroups} ( \textit{Control Groups} -- Grupos de Controle ) presente no núcleo do Linux limita e isola o uso de recursos como CPU, memória e disco de um conjunto de processos, além de organizá-los de forma hierárquica. 
%
O trabalho nessa funcionalidade começou em 2006 na Google sob a denominação de \textit{process container}. 
%
No final de 2007, seu nome foi alterado para \textit{control groups}, e o resultado foi então adicionado à versão 2.6.24 do núcleo lançado em 2008 \cite{UnixManPagesControlGroups}.

Já o \textit{Namespacing} é uma funcionalidade do núcleo do Linux usada para isolar e virtualizar recursos do sistema operacional, como identificadores de processos, acessos à rede, comunicação inter-processos e sistema de arquivos.
%
\textit{Namespacing} envolve os recursos do sistema operacional em uma abstração que faz parecer aos processos de um mesmo \textit{namespace} que estes tem sua própria instância isolada de um recurso global.
%
Desta forma, essa é a principal funcionalidade por trás da implementação de Contêineres Linux \cite{UnixManPagesNamespacing}.


Finalmente, \textit{chroot} ( \textit{Change Root} -- Troque a raiz do sistema de arquivos ) é uma funcionalidade do núcleo do Linux usada para mudar o diretório \textit{root} utilizado pelo processo que está chamando a função, bem como por todos os seus processos filhos. 
%
A chamada a \textit{chroot} altera o processo de resolução de caminhos do sistema operacional para o processo que o chamou \cite{UnixManPagesChRoot}.
%
Desta forma, pode-se instalar uma distribuição Linux secundária em uma pasta, ao invés de uma partição, e executar programas desta pasta sem perda significativa de desempenho.


\section{Forense digital e seus desafios}
\label{sec:forensedigital}


A área de forense digital (também conhecida por forense computacional) refere-se a um conjunto de técnicas de coleta e análise da interação entre humanos e computadores de forma que suas conclusões sejam aceitas em um processo legal.
%
Tal como a forense tradicional, a forense digital se baseia no princípio de Locard, estabelecido pelo médico francês Edmond Locard da seguinte forma: ``Quando um indivíduo entra em contato com outro objeto ou indivíduo, este sempre deixa vestígio deste contato'' \cite[p.~31]{Ramos:2011}.
%
De forma similar, a forense digital tem por objetivo a investigação de evidências digitais da interação entre homem e máquina, de modo a reconstruir a cadeia de eventos passados para que suas conclusões sejam validadas por terceiros e sejam aceitas em um processo legal.
 

Nas próximas subseções são detalhados os principais desafios enfrentados pela forense digital quando aplicada a infraestruturas em nuvem.

\subsection{Admissibilidade da evidência em processo legal.}
\label{sec:credibilidadeaceitabilidadeevidencia}

O termo evidência é definido por \cite[p.~1]{LuisDigitalChainOfCustody:2016} como qualquer item relavante e confiável, baseado em fatos ou dados, resultante de métodos apropriados e bem avaliados sob a perspectiva das sciências forenses que seja admitido por um juiz em um processo legal.
%
Especificamente no contexto desta proposta, evidência é qualquer informação armazenada ou transmitida no formato digital, composta de campos magnéticos e pulsos eletrônicos que podem ser coletados e analisados por meio de ferramentas adequadas \cite[p.~38]{Ramos:2011}.
%

O processo de análise forense no evento de um crime digital é descrito no Modelo Melhorado para Processo de Investigação Digital (\textit{Enhanced Digital Investigation Process Model} -- EDIPM) na forma de 4 fases \cite{GrisposChallengesCloudComputing:2012}: identificar, preservar, examinar e apresentar.
%
A fase mais pertinente a este trabalho é a de preservação da evidência, que deve ser conduzida de forma forensicamente aceitável.
%
Ou seja, deve-se coletar as evidências de forma que estas sejam aceitas em um processo legal e não tenham sua credibilidade questionada no curso do mesmo como por exemplo alterações da evidência por processos investigativos.
%
Alguns destes processos podem ser, a consolidação em um único local dos arquivos que compõem uma base de dados distribuída, a remoção da cifragem do conteúdo de um armazenamento ou a remoção da memória \textit{flash} de um dispositivo para cópia de seu conteúdo \cite{LuisDigitalChainOfCustody:2016}.
%

Para atingir o objetivo de proteger a credibilidade da evidência, o principal passo é a garantia da cadeia de custódia da mesma.
%
Cadeia de custódia é a documentação da história cronológica da evidência de modo a saber onde a evidência esteve e quem teve acesso a esta \cite[p.~21]{Ramos:2011}. 
%
Uma cadeia de custódia idealmente deve tornar visível o estado da evidência antes, durante e após a interação com o processo de investigação \cite{LuisDigitalChainOfCustody:2016}.
%
%A razão para essa preocupação é que alguns processos investigativos podem causar alteração da evidência na forma que foi coletada. 
%
%Alguns exemplos de alteração são, a consolidação em um único local dos arquivos que compõem uma base de dados distribuída, a remoção da cifragem do conteúdo de um armazenamento ou a remoção da memória \textit{flash} de um dispositivo para cópia de seu conteúdo.
%A SENASP (Secretaria Nacional de Segurança Pública) \marcos{Que sigla é essa? Cadê a referência? - Hamilton: removido, repetitivo} define cadeia de custódia como ``a sistemática de procedimentos que visa à preservação do valor probatório da prova pericial caracterizada.''


O passo seguinte é a garantia da autenticidade e da integridade da evidência.
%
Autenticidade pode ser definida como ``o processo pelo qual se pode garantir a autoria do documento eletrônico'' \cite{Ramos:2011}, ou seja, por meio do qual se não permite dúvida quanto à identificação do autor .
%
Já a integridade pode ser definida como ``o atestado da inteireza do documento eletrônico após sua transmissão, bem como apontar eventual alteração irregular de seu conteúdo'' \cite{Ramos:2011}. 
%
Caso haja dúvida acerca de um desses requisitos, uma perícia técnica pode ser convocada.
%
Nesse caso, durante a perícia é analisado o autor da evidência, ou seja, verifica-se sua fonte e se a mesma não foi alterada no processo.


Em uma infraestrutura física, a coleta de evidências pode ser feita de forma relativamente simples, bastando-se remover o recurso físico, transportar este para um laboratório e lá analisar os dados. 
%
Para limitar a exposição da evidência a manipulações indevidas, esta pode ser mantida em uma sala-cofre, à qual o acesso é controlado.
%
A reprodutibilidade do processo de coleta e a manutenção da integridade da evidência são, então, tarefas bem diretas.


Já em um cenário de computação em nuvem, especialmente as de infraestrutura auto-escalável, existe um conjunto de novos desafios. 
%
Primeiramente, em contraste com infraestruturas tradicionais, o recurso físico em princípio não pode ser removido: como os recursos são utilizados por outros usuários não relacionados à investigação, fazê-lo constituiria violação de privacidade tanto da pessoa física como da pessoa jurídica além de ser uma extrapolação dos limites de coleta concedidas pelo poder judiciário de uma região.
%
A volatilidade dos recursos também torna a verificação do seu autor um processo mais complexo, pois o recurso que gera certa evidência pode deixar de existir algum tempo depois de fazê-lo \cite{SimouCloudChlng:2014}.
%
A integridade da evidência também acaba sendo uma tarefa não trivial, pois ela precisa ser coletada, transportada e armazenada, o que caracteriza a necessidade de preservação da cadeia de custódia.
%
A violação de qualquer uma dessas características pode colocar em dúvida a credibilidade da evidência. Embora a admissibilidade da evidência em um processo legal seja uma decisão do juiz a credibilidade da mesma tem papel importante nesta decisão.
%Infelizmente, a violação de qualquer uma dessas características pode colocar em dúvida a credibilidade da evidência.


\subsection{Volume de dados para coleta}
\label{sec:volumedados}

O processo de coleta de evidências na forense digital herda suas práticas da forense tradicional, na qual isola-se cena do crime e coletam-se as evidências presentes. 
%
Transportando esse método para a forense digital, introduz-se a realização da cópia bit a bit da informação que se deseja analisar.
%
No passado, com as soluções manipulando quantidades bem menores de memória, disco e tráfego, tal prática não era considerada muito problemática. 
%
Entretanto, acesso fácil e dinâmico a recursos de armazenamento e de processamento, como máquinas virtuais, balanceadores de carga e \textit{firewalls}, aumentandou a quantidade elementos geradores de evidência \cite{WenFAAS:2013}.
%

Nas atuais soluções, aplicações e arquiteturas em nuvem, o volume de dados é consideravelmente maior \cite{QuickIncreaseVolumeImpact:2014}, \cite{Sibiya2015}.
%
Esse problema é ilustrado em \cite{QuickIncreaseVolumeImpact:2014}, na qual se discute uma investigação cuja quantidade de dados a serem analisados para fins de forense digital tomaria 6 meses.
%
Encontrar uma forma de armazenar menos informações e, assim, tornar a fase de análise mais rápida e eficiente, é um passo importante para garantir a celeridade de investigações forenses.


\subsection{Privacidade e jurisdição}
\label{sec:violacaoprivacidadejuriscdicao}

%No método tradicional de coleta de evidências para análise, isola-se o ambiente e as evidencias são removidas. \marcos{Eu acho que já li essa frase...}
%
%Transportando para a forense digital, temos a prática de remover o equipamento para realização de cópia bit a bit da evidência. 
%
%\marcos{Eu acho que já li essa frase... (sim, está repetitivo... resuma em uma frase curta, potencialmente fazendo referência à seção anterior) - Hamilton: Assim?}
%
A prática de remoção de equipamentos para coleta de evidências, mencionada na Subseção \ref{sec:volumedados}, também traz consequências negativas do ponto de vista de privacidade.
%
Mais precisamente, em soluções envolvendo infraestruturas físicas, tal prática não costuma trazer grandes problemas porque os objetos ou indivíduos sob investigação normalmente estão diretamente relacionados ao equipamento removido.
%
Nas soluções em nuvem, entretanto, tal prática não é recomendada porque o recurso físico é compartilhado por vários usuários, inclusive indivíduos não envolvidos na investigação.
%
Logo, remover tais recursos configuraria violação de privacidade \cite{BashAdvInForensics:2015}.
%
Como um complicador adicional, o fato de os dados potencialmente não estarem armazenados no mesmo território em que a investigação é realizada acaba demandando acordos de cooperação jurídica entre as partes, o que nem sempre é possível \cite{SimouCloudChlng:2014}.


%O ''State of the Cloud Report'', relatório produzido pela empresa Right Scale \cite{RightScale2018}, cita nas páginas 21 e 22 que os maiores desafios na adoção de nuvem são, \textit{gerenciamento de custo} e \textit{segurança}.
%
Embora privacidade e jurisdição sejam aspectos distintos e possuidores de suas particularidades, neste documento estes são discutidos em conjunto pois em busca de otimização de custos as empresas tendem a buscar regiões mais baratas e compartilhamento do recurso físico.
%
Neste cenário, encontrar uma forma de coletar a evidência sem violar jurisdição e privacidade ganham importância.


\subsection{Coleta de evidências de memória volátil de máquinas em nuvem}
\label{sec:forensenuvem}

A prática de armazenar histórico de tráfego de rede e alterações de dados armazenados em disco já é bem difundida na comunidade forense.
%
Por outro lado, a memória volátil de computadores não costuma receber o mesmo tratamento: suas alterações quase nunca são armazenadas, seja por questões de desempenho ou por simples praticidade, dado a reduzida sobrevida dessas informações.
%
Infelizmente, isso acaba dificultando a análise de uma classe específica de ataques, conhecidos como injeção de código em memória \cite{CaseMemoryForensics:2014}. 
%
Assim, quando usados contra uma arquitetura em nuvem, tais ataques não deixam rastros quando recursos de processamento virtuais são desativados e sua memória é liberada \cite{VomelMemoryAcquisition:2013,CaseMemoryForensics:2014}.
%
Em particular, têm especial interesse quatro tipos particulares dessa família de ameaças \cite{CaseMemoryForensics:2014}:


\begin{itemize}
 \item \textbf{Injeção remota de bibliotecas}: Um processo malicioso força o processo alvo a carregar uma biblioteca em seu espaço de memória.
 %
 Como resultado, o código da biblioteca carregada executa com os mesmos privilégios do executável em que ela foi injetada. 
 %
 Tal estratégia, comumente usada para instalar códigos maliciosos, pode fazer com que uma biblioteca maliciosa armazenada no sistema seja distribuída por vários processos de uma mesma máquina, dificultando sua remoção \cite{MillerRemoteLibraryInjection:2004}.
 
 \item \textbf{Inline Hooking}: Um processo malicioso escreve código como uma sequência de bytes diretamente no espaço de memória de um processo alvo, e então força este último a executar o código injetado. 
 %
 O código pode ser, por exemplo, um \textit{shell script}.
 

 \item \textbf{Injeção reflexiva de biblioteca}: Um processo malicioso acessa diretamente a memória do processo alvo, inserindo nela o código de uma biblioteca na forma de uma sequência de bytes, e então força o processo a executar essa biblioteca. 
 %
 Nessa forma de ataque, a biblioteca maliciosa não existe fisicamente; isso torna tal estratégia de injeção de código potencialmente mais atrativa, pois o carregamento da biblioteca não é registrado no sistema operacional (SO), dificultando a detecção do ataque \cite{FewerReflectiveLibraryInject:2008}.
 
 \item \textbf{Injeção de processo vazio}: Um processo malicioso dispara uma instância de um processo legítimo no estado ``suspenso''; a área do executável é então liberada e realocada com código malicioso.
\end{itemize}


\section{Forense digital e gestão de incidentes}
\label{sec:forenseeincidentes}

%
Gestão de incidentes é uma estratégia para reduzir os riscos à confidencialidade, integridade e disponibilidade de ativos de uma organização, bem como minimizar sua perda.
%
As fases relevantes de uma estratégia de gestão de incidentes são \cite{AdhiantoFasesGestaoIncidente:2010}, \cite{CichonskiNIST:2012}: 


\begin{itemize}
%
\item \textbf{Preparação}: Onde se organiza o ambiente para minimizar impactos de um incidente, nesta fase também é definido o time de resposta a incidente que tem como responsabilidade determinar o que ocorreu e quais ações devem ser tomadas.
%
\item \textbf{Detecção}: Tem como objetivo minimizar o risco das ameaças que não foram antecipadas. A fase de detecção começa quando alguma atividade suspeita é detectada por algum serviço automático como um sistema de detecção de ameaças ou por intervenção humana como o de um responsável pela análise de risco.
%
\item \textbf{Resposta ao incidente}: Contém as fases de \textit{contenção}, \textit{erradicação} e \textit{recuperação}. Não existe um procedimento que atenda a todas os cenários, os procedimentos são feitos de acordo com a natureza e as necessidades do negócio.
%
\item \textbf{Pós incidente}: Última fase da gestão de incidentes, sua principal preocupação é coleta de informações das três fases anteriores para alimentar um processo de aprendizado visando evitar futuros incidentes semelhantes.

\end{itemize}

A gestão de incidentes tem o foco em conter falhas de segurança. 
%
Considerações como coleta de evidências costuma ser secundário \cite{AdhiantoFasesGestaoIncidente:2010}. 
%
A oportunidade de se aplicar processos de coleta forenses em gestão de incidente já foi levantado por \cite{AdhiantoIncidentHandlingForensic:2016} uma vez que ambos compartilham de um ferramental em comum.
%

%
Ferramental e técnicas forenses estão se tornando úteis não só para geração de evidências para processos jurídicos mas também para ajudar na reconstrução de eventos.
%
Incorporar práticas forenses na estratégia de gestão de incidente, ajuda na geração das evidências necessárias para eventual processo jurídico como também ajuda a proteger as evidencias de dano como resultado das fase de resposta ao incidente.





