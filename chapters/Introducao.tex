% ----------------------------------------------------------
% Introdução
% ----------------------------------------------------------
\chapter{Introdução}

\section{Problema de pesquisa}

%==== CONTEXTO GERAL: Nuvem e volatilidade de VMs ====
%
Técnicas de virtualização, replicação de serviços e compartilhamento de recursos entre múltiplos usuários (multi-inquilinato) proveem a nuvens computacionais uma elevada escalabilidade \cite{MorsyCloudSecurity:2010}.
%
Ao mesmo tempo, tais mecanismos também criam uma elevada volatilidade dos recursos virtuais que executam aplicações em nuvem.
%
Afinal, quando submetida a uma carga elevada, uma aplicação hospedada na nuvem pode criar clones das máquinas virtuais (\textit{virtual machines} -- VMs) que a hospedam e balancear a carga entre elas, de modo a atender à demanda sem prejuízos na qualidade do serviço oferecido. 
%
Após esse pico, as máquinas que foram clonadas são normalmente desativadas, seus recursos liberados e o sistema retorna à capacidade anterior, evitando-se custos desnecessários.


%==== CONTEXTO ESPECÍFICO + PROBLEMA GERAL: Forense na nuvem vs. volatilidade + multitenancy + multidomains ====
%
Embora interessante do ponto de vista de eficiência e custos, do ponto de vista forense a volatilidade da nuvem traz problemas em caso de ataques.
%
Por exemplo, caso uma das instâncias de processamento virtuais criadas temporariamente seja alvo de ameaças que atuam diretamente na sua memória, sem deixar rastros em discos (e.g., arquivos de \textit{log}), as evidências desse evento podem ser completamente perdidas após elas serem desativadas e terem seus recursos liberados.
%
Essa dificuldade é ainda agravada por aspectos como multi-inquilinato e multi-jurisdição típicas de soluções em nuvem \cite{BashAdvInForensics:2015}.
%
Especificamente, o aspecto multi-inquilino dificulta a obtenção do \textit{hardware} que executa as aplicações de interesse, pois, como ele é compartilhado por vários usuários, removê-los para análise poderia levar a uma violação de privacidade dos usuários não relacionados à investigação. 
%
Já a natureza distribuída da nuvem pode levar à alocação de informações relevantes à investigação em vários países, dificultando a obtenção das mesmas em especial quando não existem acordos de cooperação entre as entidades envolvidas \cite{DykstraAcquiringForIAAS:2012}.
%
Combinadas, tais características dificultam a coleta de evidências com a credibilidade necessária para que elas possam ser usadas em processos legais,  o que exige o respeito à privacidade, à jurisdição e à cadeia de custódia, bem como a reprodutibilidade do processo de coleta \cite{RahmanLiveForensicsTechniques:2015}.

%==== O QUE EXISTE E PORQUE NÃO É SUFICIENTE: ??? ====
%
Embora existam soluções na literatura que abordam a coleta de informações de nuvem com o propósito de análise forense, a grande maioria delas aborda a coleta, o transporte e o armazenamento de forma isolada.
%
Por exemplo, trabalhos como \cite{DykstraFROST:2013} e \cite{ReichertAutoAcquisition:2015} tratam de fatores como multi-inquilinato e multi-jurisdição, discutindo formas de coleta e preservação da evidência fora da nuvem.
%
Já estudos como \cite{GeorgeDF2CE:2012} se concentram na análise forense para a coleta de evidência de máquinas virtuais enquanto elas estão em execução, enquanto trabalhos como \cite{SangLogApproach:2013} abordam a questão de processos de garantia de cadeia de custódia em ambientes de nuvem para transporte da evidência.
%
%abordam os problemas descritos anteriormente \marcosT{QUAIS, CARA PÁLIDA? VOCÊ LISTOU 4 TIPOS DE ATAQUE E DEIXOU UM PROBLEMA GERAL: COMO O LEITOR VAI SABER QUAIS EXATAMENTE SÃO ABORDADOS? Seja mais preciso: clareza acima de tudo!!!} de forma isolada.
%
%Alguns propõem soluções para o os fatores multi-inquilino e multi-jurisdição, outros abordam apenas a coleta de evidência de máquinas virtuais e por fim temos as que descrevem apenas os processos de garantia de cadeia de custodia. \marcosT{Er... você não me deu sequer um exemplo, então eu sou obrigado a acreditar em você sem que você apresente argumentos... lembre-se que revisores são seres amargos e cruéis, que não acreditam em nada, então melhor não arriscar e dar exemplos claros.}
%
Por outro lado, não foram identificadas na literatura propostas que (1) descrevam como o dado é coletado e armazenado observando a cadeia de custódia, e (2) visem garantir que, mesmo que um recurso virtualizado (e.g., uma VM) seja desalocada, haja condições de se reproduzir o processo de coleta de evidências.

\section{Objetivos}

%==== O QUE FAZEMOS: Ataques de injeção ====
%
O presente trabalho visa suplantar tais limitações por meio de uma proposta que tem como focos (1) a reprodutibilidade do processo de coleta, (2) o estabelecimento de vínculo entre a evidência coletada e sua origem, (3) a preservação da jurisdição e da privacidade dos não envolvidos na investigação e (4) a garantia de custódia da evidência.
%
Em suma, a solução descrita provê uma forma de correlacionar evidências e sua origem virtual, permitindo transportar e armazenar tais dados de modo a preservar sua credibilidade.
%
Para isso, a proposta supõe que o sistema sendo monitorado é executado dentro de um contêiner em nuvem. 
%
O foco da solução em contêiner se justifica pelo crescimento da adoção de conteinerização nos últimos anos, bem como pela previsão de que este será o modelo de implementação mais usado em aplicações futuras \cite{PiraghajContainerCloudComputing:2016}.
%
A solução tem como alvo específico ataques de injeção de código \cite{CaseMemoryForensics:2014}, pois estes, quando usados contra uma arquitetura em nuvem, não deixam rastros quando recursos de processamento virtuais são desativados e sua memória é liberada \cite{VomelMemoryAcquisition:2013}, \cite{CaseMemoryForensics:2014}.
%
Em particular, têm especial interesse quatro tipos específicos dessa família de ameaças \cite{CaseMemoryForensics:2014}:


\begin{itemize}
 \item \textbf{Injeção remota de bibliotecas}: Um processo malicioso força o processo alvo a carregar uma biblioteca em seu espaço de memória.
 %
 Como resultado, o código da biblioteca carregada executa com os mesmos privilégios do executável em que ela foi injetada. 
 %
 Esta estratégia, comumente usada para instalar malwares, pode fazer com que uma biblioteca maliciosa armazenada no sistema seja distribuída por vários processos de uma mesma máquina, dificultando sua remoção \cite{MillerRemoteLibraryInjection:2004}.
 %
 \item \textbf{Inline Hooking}: Um processo malicioso escreve código como uma sequência de bytes diretamente no espaço de memória de um processo alvo, e então força este último a executar o código injetado. 
 %
 O código pode ser, por exemplo, um \textit{script} de \textit{shell}.
 %
 \item \textbf{Injeção reflexiva de biblioteca}: Um processo malicioso acessa diretamente a memória do processo alvo, inserindo nela o código de uma biblioteca na forma de uma sequência de bytes, e então força o processo a executar essa biblioteca. 
 %
 Nessa forma de ataque, a biblioteca maliciosa não existe fisicamente; isso torna tal estratégia de injeção de código potencialmente mais atrativa, pois o carregamento da biblioteca não é registrado no sistema operacional (SO), dificultando a detecção do ataque \cite{FewerReflectiveLibraryInject:2008}.
 %
 \item \textbf{Injeção de processo vazio}: Um processo malicioso dispara uma instância de um processo legítimo no estado ``suspenso''; a área do executável é então liberada e realocada com código malicioso.
\end{itemize}

\section{Justificatica}

Uma nuvem computacional é um modelo de infraestrutura no qual recursos compartilhados em quantidade configurável, acessíveis via rede, são alocados e desalocados com esforço mínimo de gerenciamento por parte de um provedor de serviços. \cite{NIST2011}
%A infraestrutura é composta de máquinas físicas contendo cada uma um número variável de máquinas virtuais que implementam este serviço \cite{Sousa_Computacao_Nuvem:2009}. 
%
Há três modelos principais de comercialização de uso da nuvem \cite{NIST2011}: \textit{software} como serviço (\textit{Software as a Service} -- SaaS), na qual se provê o \textit{software} que será usado pelo cliente; plataforma como serviço (\textit{Platform as a Service} -- PaaS), na qual se provê o ambiente para que o cliente desenvolva, teste e execute seu \textit{software}; e, o tipo mais pertinente para este trabalho, Infraestrutura como serviço (\textit{Infrastructure as a Service} -- IaaS), na qual são fornecidos recursos computacionais básicos, como processamento e memória, em geral de forma virtualizada.


A virtualização de recursos na nuvem, embora tradicionalmente feita por meio de máquinas virtuais, vêm sendo crescentemente feita também na forma de contêineres.
%
De fato, segundo o ``Container Market Adoption Survey 2016'', realizado pelas empresas DevOps.com (https://devops.com/) e ClusterHQ (https://clusterhq.com) com 235 empresas que têm desenvolvimento de software como sua atividade fim ou como suporte à atividade fim, 76\% dos respondentes utilizam contêineres para melhorar a eficiência do processo de desenvolvimento e em suas arquiteturas de micro-serviços em nuvem.
%
Diferentemente de máquinas virtuais, que envolvem a criação de um \textit{hardware} virtual e de um sistema operacional (SO) acima do sistema nativo e que opera independente deste, a virtualização com contêineres é feita no nível do SO nativo, tem uma implementação mais simples eliminado camadas entre o aplicativo executado e o \textit{hardware} físico.
%
Uma tecnologia bastante utilizada para esse propósito são Contêineres Linux (LXC) \cite{Linuxcontainers.org2015}, que aproveitam-se de funcionalidades como cgroups e namespacing do kernel do Linux para auxiliar no gerenciamento e isolamento de recursos virtuais.


\section{Método de pesquisa}

Como eu vou chegar a este resultado.

\section{Organização de documento}

O bla bla bla de sempre