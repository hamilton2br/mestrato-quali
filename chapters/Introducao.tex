% ----------------------------------------------------------
% Introdução
% ----------------------------------------------------------
\chapter{Introdução}
\label{chp:intro}

%==== CONTEXTO GERAL: Nuvem e volatilidade de VMs ====
%
Técnicas de virtualização, replicação de serviços e compartilhamento de recursos entre múltiplos usuários (multi-inquilinato) provêem a nuvens computacionais uma elevada escalabilidade \cite{MorsyCloudSecurity:2010}.
%
Ao mesmo tempo, tais mecanismos também tornam voláteis os recursos virtuais que executam aplicações em nuvem.
%
Afinal, quando submetida a uma carga elevada, uma aplicação hospedada na nuvem pode criar instâncias das máquinas virtuais que a hospedam e balancear a carga entre elas.
%
Isso permite ao sistema em nuvem ter elasticidade isto é, redimencionar seus recursos de acordo com a demanda os usuários da aplicação. 
%
Após esse pico, as máquinas que foram instânciadas são normalmente desativadas, seus recursos liberados e o sistema retorna à capacidade anterior, evitando-se desperdício de recursos.


%==== CONTEXTO ESPECÍFICO + PROBLEMA GERAL: Forense na nuvem vs. volatilidade + multitenancy + multidomains ====
%
Embora interessante do ponto de vista de eficiência e custos, do ponto de vista forense a volatilidade da nuvem traz problemas em caso de ataques.
%
Por exemplo, pode-se supor um cenário em que uma das instâncias de processamento virtuais criadas temporariamente seja alvo de ameaças que atuam diretamente na sua memória, sem deixar rastros no armazenamento. (e.g., em arquivos de \textit{log}).
%
Nesse caso, as evidências desse evento podem ser completamente perdidas após as instâncias em questão serem desativadas e terem seus recursos liberados.
%
Essa dificuldade é ainda agravada por aspectos como multi-inquilinato e multi-jurisdição, típicas de soluções em nuvem pública ou híbrida \cite{BashAdvInForensics:2015}.
%
Especificamente, o multi-inquilinato dificulta a obtenção do \textit{hardware} que executa as aplicações de interesse.
%
Afinal, como este é compartilhado por vários usuários, removê-los para análise poderia levar a uma violação de privacidade dos usuários não relacionados à investigação. 
%
Já a natureza distribuída da nuvem pode levar à alocação de informações relevantes em vários países, dificultando a sua obtenção, em especial quando não existem acordos de cooperação entre as entidades envolvidas \cite{DykstraAcquiringForIAAS:2012}.
%
Combinadas, tais características limitam a coleta de evidências com a credibilidade necessária para que elas possam ser aceitas em processos legais \cite{RahmanLiveForensicsTechniques:2015}, que exigem o respeito à privacidade, à jurisdição e à cadeia de custódia, bem como a reprodutibilidade do processo de coleta.
\section{Motivação}
\label{sec:intro-motiv}


%==== O QUE EXISTE E PORQUE NÃO É SUFICIENTE: ??? ====
%
Embora existam soluções na literatura que abordam a coleta de informações de nuvem com o propósito de análise forense, a grande maioria delas aborda a coleta, o transporte e o armazenamento de forma isolada.
%
Por exemplo, trabalhos como \cite{DykstraFROST:2013} e \cite{ReichertAutoAcquisition:2015} tratam de fatores como multi-inquilinato e multi-jurisdição, discutindo formas de coleta e preservação da evidência fora da nuvem.
%
Já estudos como \cite{GeorgeDF2CE:2012} se concentram na análise forense para a coleta de evidência de máquinas virtuais em tempo de execução, enquanto trabalhos como \cite{SangLogApproach:2013} abordam processos de garantia da cadeia de custódia em ambientes de nuvem para transporte da evidência.
%
%abordam os problemas descritos anteriormente \marcosT{QUAIS, CARA PÁLIDA? VOCÊ LISTOU 4 TIPOS DE ATAQUE E DEIXOU UM PROBLEMA GERAL: COMO O LEITOR VAI SABER QUAIS EXATAMENTE SÃO ABORDADOS? Seja mais preciso: clareza acima de tudo!!!} de forma isolada.
%
%Alguns propõem soluções para o os fatores multi-inquilino e multi-jurisdição, outros abordam apenas a coleta de evidência de máquinas virtuais e por fim temos as que descrevem apenas os processos de garantia de cadeia de custodia. \marcosT{Er... você não me deu sequer um exemplo, então eu sou obrigado a acreditar em você sem que você apresente argumentos... lembre-se que revisores são seres amargos e cruéis, que não acreditam em nada, então melhor não arriscar e dar exemplos claros.}
%
Por outro lado, não foram identificadas na literatura propostas que (1) descrevam como o dado é coletado e armazenado observando a cadeia de custódia, e (2) visem garantir que, mesmo que um recurso virtualizado (e.g., uma máquina virtual) seja desalocado, haja condições de se reproduzir o processo de coleta de evidências.

\section{Objetivos}
\label{sec:intro-objetivos}

%==== O QUE FAZEMOS: Ataques de injeção ====
%
O presente trabalho visa suplantar as limitações identificadas na literatura com relação à capacidade de coleta de evidências de aplicações em nuvem.
%
Mais precisamente, por meio de uma proposta que tem como focos (1) a reprodutibilidade do processo de coleta, (2) o estabelecimento de um vínculo entre a evidência coletada e sua origem, (3) a preservação da jurisdição e da privacidade dos não envolvidos na investigação e (4) a garantia de custódia da evidência.
%
Desta forma, a solução buscada deve prover uma forma de obter dados relevantes para investigações ao mesmo tempo que permita transportar e armazenar tais dados preservando sua credibilidade.
%
Para isso, a proposta supõe que o sistema sendo monitorado é executado dentro de um ambiente virtualizado em nuvem.
%
%
A presente proposta tem como foco prover uma solução técnica que atenda as necessidades jurídicas presentes hoje, não faz parte dos objetivos deste trabalho propor alterações na legislação ou nos procedimentos legais vigentes.


\section{Justificativa}
\label{sec:intro-justificativa}

O crescente volume de informações que as soluções em nuvem armazenam e trafegam atualmente, bem como os aspectos de multi-inquilinato e a multi-jurisdição dos provedores de nuvem, estão entre os principais obstáculos enfrentados pelos investigadores forenses. \cite{QuickIncreaseVolumeImpact:2014} \cite{BashAdvInForensics:2015}.
%
Desta forma, a criação de soluções que permitam melhorar o processo de investigação forense nesse cenário tem relevância não apenas teórica, mas também prática.


Ao mesmo tempo, virtualização tem sido amplamente adotada por empresas das mais diversas áreas. 
%
Segundo o ''State of the Cloud Report'', relatório produzido pela empresa Right Scale \cite{RightScale2018}, 96\% das organizações entrevistadas estão utilizando ou experimentando soluções em nuvem no modelo IaaS ( \textit{Infrastructure as a Service} -- Infraestrutura como um Serviço ).
%
Embora a virtualização de recursos na nuvem seja tradicionalmente feita por meio de máquinas virtuais, tem ganhado importância também a tecnologia de contêineres \cite{Diamanti2019}, uma forma de virtualização bastante leve e flexível.
%
De fato, conforme o ``Container Market Adoption Survey 2018'' \cite{Portworx2018}, realizado com 519 empresas que têm desenvolvimento de software como sua atividade fim ou como suporte à atividade fim, 83,5\% dos respondentes utilizam contêineres em ambiente produtivo em suas arquiteturas de microsserviços em nuvem.
%

Por fim, gestão de incidentes, que cada vez mais compartilha processos em comum com forense digital principalmente na fase de detecção e análise, pode se beneficiar de uma solução de coleta e preservação de evidências. Ter uma solução que esteja pronta na fase de \textit{preparação ao incidente} e seja capaz de preservar as evidências para a fase de \textit{pós-incidente}, permite que na fase de \textit{detecção e análise} não seja preciso se preocupar com perda parcial ou total de evidências.


Essa tendência motiva a construção de soluções que levem em consideração as particularidades inerentes aos recursos disponíveis em nuvens computacionais.
%Essa tendência motiva a construção de soluções que levem em consideração as particularidades inerentes a contêineres computacionais, e que sejam capazes de tirar vantagem de suas características. reescrevi para dar menos ênfase a contêiner


\section{Organização do documento}
\label{sec:intro-organizacao}

O restante deste documento está organizado conforme os seguintes capítulos.

O Capítulo \ref{chp:fundamentação} apresenta o ambiente em que este trabalho está inserido, as tecnologias envolvidas e os desafios a serem superados. 

O Capítulo \ref{chp:revisão} analisa os trabalhos relacionados a forense de memória em nuvem. 

O Capítulo \ref{chp:proposta} apresenta as atividades realizadas como parte do trabalho de pesquisa aqui apresentado, bem como suas conclusões.